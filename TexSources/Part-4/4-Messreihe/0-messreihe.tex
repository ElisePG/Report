

Die zweite Messreihe fand in zwei Räumen des Forschungskolleg Siegens (FoKoS) statt. 
Die Messreihe wurde in den beiden Wochen vom  14.01.19 bis 25.01.19 statt. 
Insgesamt wurden Daten von Daten von 97 Personen erhalten. 
Diese deutliche Steigerung der Teilnehmerzahl im Vergleich zur ersten Messreihe (hier waren es 22 Teilnehmer) wurde durch eine Auszahlung von 10 Euro erreicht. 
Bei der vorherigen Messreihe wurde an die Teilnehmer keine Aufwandsentschädigung für die Teilnahme geleistet. 
Eine genaue Auswertung der Daten folgt in späteren Kapiteln. Von der reinen Durchführung her lässt sich aber sagen, dass es mit nur einer Ausnahme keine Probleme bezüglich VR gab (Schwindel).
Die Messungen wurden mit der an vorheriger Stelle schon beschriebenem dritten Prototypen gemacht Abb.1 . Das Messsystem bestand aus dem dritten Prototypen, der in einem extra hierfür gefertigten Gehäuse verbaut wurde. Zusätzlich wurde die in Realisierung beschriebene Maske verwendet. 