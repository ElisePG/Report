\todo[inline]{Verantwortlich: Artur}


Wegen anderer Studien, wurden jeweils DNNs für ein zwei-Klassen- und ein drei-Klassen-Mustererkennungsproblem getestet.
Zudem wurde zum einem Probanden-abhängige und zum anderen Probanden-unabhängige Datensätze verwendet, um z.B. Overfitting feststellen zu können.
Im Folgendem sind die F1-Scores für Errenung und Wertigkeit (entsprechend dem Circumplex-Modell) vorgestellt, wobei MLP die besten Ergebnisse erzielten. \\ 


\begin{table}[H] \centering
\begin{tabular}{|c|c|c|}
\hline
\textbf{Erregung F1-Score (\%)} & \textbf{Probanden-abhängig} & \textbf{Probanden-unabhängig} \\ \hline
\textbf{2 Klassen} & 90,57\% & 58,35\% \\ \hline
\textbf{3 Klassen} & 83,08\% & 42,65\% \\ \hline
\end{tabular} \vspace{0.2cm}
\caption{ F1-Score der DNNs für Erregung. } \end{table}



\begin{table}[H] \centering
\begin{tabular}{|c|c|c|}
\hline
\textbf{Wertigkeit F1-Score (\%)} & \textbf{Probanden-abhängig} & \textbf{Probanden-unabhängig} \\ \hline
\textbf{2 Klassen} & 88,21\% & 44,07\% \\ \hline
\textbf{3 Klassen} & 84,72\% & 21,86\% \\ \hline
\end{tabular} \vspace{0.2cm}
\caption{ F1-Score der DNNs für Wertigkeit. } \end{table}


Es ist klar zu erkennen, dass die Probanden-anhängige Ergebnisse relativ gut sind und die Probanden-unanhängige immer sanken, d.h. relativ schlecht sind. 
Wie bei dem ersten Datensatz, findet hier wahrscheinlich auch Overfitting bei den Probanden-anhängigen Datensatz statt.
Auch hier wurden ähnliche t-SNE Plots wie in Kapitel \ref{analyse-subsec} erstellt, die diese Vermutung bestätigen. 
Bei dem Probanden-unanhängigen Datensatz liefern DNNs keine gute Ergebnisse, wobei die Performance für Erregung klar besser ist als für Wertigkeit. 
