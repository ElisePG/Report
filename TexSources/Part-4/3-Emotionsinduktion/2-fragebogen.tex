\subsection{Fragebogen} \label{fragebogen-4}


\todo[inline]{Verantwortlich: Boris}

Für dieses Prototyp wurden zwei Arten von Fragebögen benutzt. 
Der erste Typ ist sehr ähnlich mit dem von der zweite Prototyp. Es enthält in dem informativen Teil einen Text, wo es beschrieben wird, wie der Fragebogen ausgefüllt werden soll. Der andere Teil besteht aus vier Dropdown-Boxen von dreizehn Optionen, die zwölf verschiedene Emotionen und einen als null oder neutral geltenden Zustand enthalten. Jede Dropdown-Box entspricht ein Viertelzeit der Szenario. Es soll zwischen die Optionen jeder Dropdown-Box gewählt werden, welche Emotion es am stärksten empfindet wird, je nachdem, wann man es fühlte, d.h. ob es das erste, zweite, dritte oder letzte Quartal der Zeit des Videos war, um die Emotion zu bewältigen. Unten gibt es ein Button wo man drücken kann, wenn man fertig ist. Allerdings hat man auch die Möglichkeit seine Wahl zu ändern, auch wenn man sich schon im nächsten Schritt befindet, indem man in diesem nächsten Schritt auf den Zurück-Button drückt und die gewünschten Änderungen vornimmt. Hierfür wird ein Widget Blueprint erstellt, die vier Dropdown-Boxen mit dem gewünschten Anzahl an Optionen hinzugefügt und die Labels von der unterschiedlichen Optionen der Dropdown-Boxen definiert. Ein Button ``next'' wird auch erstellt um zum nächsten Fragebogen zu navigieren. Es wird auch eine zwischen Speicherungsfunktion in ein anderes Skript definiert, die hier aufgerufen wird, um die Änderung auch nach das drücken von dem ``next'' Button zu Speichern. Dabei wird vier Variablen definiert und die Werte von der gewählten Optionen werden ihnen zugewiesen. \\

(Bild für Fragebogen) \\


Der zweite Typ ähnelt dem berühmten Modell von James Russels ``circumplex'' \cite{russel_1980}. Es ist ein klassisches Modell mit einer kreisförmigen Struktur, die auf zwei senkrechten Diagonalen ruht. Die vertikale Achse, die die Erregung darstellt, und die horizontale Achse, die die Valenz darstellt. Das Zentrum des Kreises stellt eine neutrale Valenz und ein mittleres Erregungsniveau dar. Andere Emotionen werden auf jeder Ebene des Kreises dargestellt.  Hier wird ein weniger bekanntes Modell verwendet, das ``Self Assessment Manikin'' (SAM). Es besteht aus drei Reihen mit je fünf Piktogrammen. Diese Piktogramme stellen den Zustand eines Gesichts nach verschiedenen Arten von Emotionen dar. So repräsentiert der erste Bereich die Wertigkeit, der zweite die Erregung und der dritte die Dominanz. Eine Erklärung zu jedem dieser Begriffe ist ebenfalls neben dem Fragebogen enthalten, um die Testpersonen über diese Wörter aufzuklären.  Bei jeder Avatar und in der Mitte jeder der beiden Avatar befindet sich ein Checkbox.  So muss man für jede Zeile das Checkbox auswählen, das ihrem emotionalen Zustand am besten entspricht. Man kann nur ein Checkbox pro Zeile markieren und man hat auch die Möglichkeit wie bei dem ersten Model seine Wahl zu ändern. Es kann einfach mit Branch-Bedingungen realisieren werden.  Diese werden auch in eine Widget Blueprint wie für das erste Modell gemacht. Es gibt auch wieder die zwischen Speicherungsfunktion und das Button ``next''. Was neues hier kommt ist das Button ``back'' um wieder zum ersten Fragebogen zu navigieren. 

(Bild für circumplex) \\