\subsubsection{Frustration} \label{frust-4}

\todo[inline]{Verantwortlich: Meryem}



Die Frustration stellt einen negativen Zustand des Menschen dar, welcher mehrere Indikatoren haben kann. Dieser Zustand kann sowohl eine Gefühlslage als auch eine Folge vorhergehender Emotionen sein.
Im Allgemeinen setzt die Frustration ein, wenn Misserfolgserlebnisse sowie Versagungs- und Enttäuschungserlebnisse vorliegen. Ursachen dafür können in der Person selber und/ oder in Umwelteinflüssen liegen. 
Frustration wird ausgelöst, wenn die Person eine wirkliche Benachteiligung erleidet oder sie als diese wahrnimmt. Eine empfundene Ungerechtigkeit resultierend aus Erwartungen der Umwelt an die Person sowie Erwartungen, die an sich selbst gerichtet sind, können den Menschen frustrien. \\

Für das Frustrationsexpermiment der Studie soll auf das Auslösen der Misserfolgserlebnisse sowie eine empfundene Ungerechtigkeit zurückgegriffen werden.
Dem Probanden wird die Aufgabe gestellt, in eine VR-Umgebung das Spiel ``heißer Draht'' zu spielen. In diesem Spiel besteht die Aufgabe darin, den Ring, welcher sich in der Hand des Spieler befindet, von dem Startpunkt zum Endpunkt eines Drahts zu befördern. Der Ring darf währenddessen den Draht nicht berühren.
Dieses Spiel ist durch die geschwungene Form des Drahts schwierig und fordert daher viel Ruhe und Geschick. 
Dem Probanden wird eine Zeit vorgegeben, in welcher dieser von dem Startpunkt zum Endpunkt gelangen muss. 
Der Faktor des Zeitdrucks kann eine Stressreaktion auslösen. Da das Spiel jedoch den Ehrgeiz erwecken kann, kann es dazu führen, dass es nicht zu der erhofften Frustration kommt. Um diese Emotion trotzdem einleiten zu können, wird das Spiel an einigen Stellen so prepariert, dass der Proband durch Spielfehler den Draht berührt und erneut von dem Startpunkt aus starten muss. Die empfundene Ungerechtigkeit und Ohnmächtigkeit des Probanden soll als Indikator der Frustration dienen. 