\subsubsection{Langeweile} \label{langeweile-4}



\todo[inline]{Verantwortlich: Boris}


In diesem Prototyp wird nur das "Pec-Turning“ Spiel als Szenario benutzt, da man die Zeit des Experiments reduzieren wollte. Allerdings gib es ein kleines Unterschied mit dem Spiel von dem zweiten Prototyp. Hier wird die grüne kreisförmige Scheibe durch den Proband selbst in Bewegung gebracht. Nach jeder Bewegung des Kreis muss man mindesten fünften Sekunde warten bis die nächste Bewegung möglich ist. Es wird auch jederzeit gezeigt, wenn man die Scheibe drehen kann und wenn man warten muss. Das Szenario dauert wieder fünf Minuten. \\

(Bild für Langweile in VR mit Textbeschreibung) \\