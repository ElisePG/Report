\subsubsection{Langeweile} \label{langeweile-4}



\todo[inline]{Verantwortlich: Boris}


In diesem Prototyp wird nur das ``Pec-Turning'' Spiel als Szenario benutzt, da man die Zeit des Experiments reduzieren wollte. Allerdings gib es ein kleines Unterschied mit dem Spiel von dem zweiten Prototyp. Hier wird die grüne kreisförmige Scheibe durch den Proband selbst in Bewegung gebracht. Nach jeder Bewegung des Kreis muss man mindesten fünften Sekunde warten bis die nächste Bewegung möglich ist. 
 Um dies Szenario in Unreal Engine zu verwirklichen habe ich erstmal ein Widget Blueprint erstellt. Dann habe ich vier mal das Bild von dem Spiel genau an der gleiche Stelle und in der gleiche Große in der Widget hinzugefügt. Das erste Bild ist in der Ausgangsposition. Dann wird jedes der nächsten Bilder eine Drehung des vorherigen Bildes um neunzig Grad sein. Es wird von Anfang an das Ausgangsbild auf sichtbar gesetzt und die anderen auf unsichtbar. Danach habe ich einen transparenten Button immer auf die Bilder der Scheiben eingefügt, um das Wechsel von Bilder zu steuern. Zur Programmierung erstelle ich eine Funktion mit Timeout und Branch-Bedingung und der Algorithmus wird so rekursiv definiert:  Wenn das Button gedrückt wird, wird das nächste Bild auf sichtbar (``visible'') gesetzt und die anderen Bilder auf unsichtbar (``unvisible''), dann wird ein Timeout gesetzt und eine Branch-Bedingung soll Prüfen ob das Timeout durch ist. Sollte man vom Ende des Timeouts der Button drücken, so sollte nichts passieren. Wenn das Timeout fertig ist, soll das Szenario sich wiederholen. Es wird auch jede zeit durch einen Text in einem Textfeld gezeigt, wenn man die Scheibe drehen kann und wenn man warten muss. Ein letztes Timeout wird hinzugefügt, um zu prüfen, dass das Szenario ein gewünschtes Zeit dauert. \\

(Bild für Langweile in VR mit Textbeschreibung)