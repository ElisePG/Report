\subsection{Ablauf} \label{ablauf-subsubsec}


\todo[inline]{Verantwortlich: Minas}


Der Ablauf der Emotionsinduktion lief wie folgt ab:


\begin{itemize}
\item[1.] Einführung %$\rightarrow$ Kapitel 7.3.1
\item[2.] WarmUp %$\rightarrow$ Kapitel 7.3.2
\item[3.] Glücks-Szenario %$\rightarrow$ Kapitel 7.3.3
\item[4.] Langeweile-Szenario %$\rightarrow$ Kapitel 7.3.2
\item[5.] Frustrations-Szenario
\item[6.] Schlussworte
\end{itemize}


Die Emotionsinduktion fing mit einer Einführung an, welche das Projekt vorstellt.
Es sollte ein breites Spektrum von Probanden angesprochen werden. 
Viele von ihnen hatten jedoch noch keinen Kontakt zu einer VR-Anwendung oder zur VR-Hardware. 
Um die Emotionsinduktion dennoch erfolgreich abzuschließen, wurde ein WarmUp-Szenario erstellt. 
In diesem sollten die Probanden einen ersten Einblick in die VR-Anwendung bzw. VR-Hardware bekommen. 
Das WarmUp-Szenario bestand aus einem 360 Grad Bild, worauf ein kleiner Fragebogen erschien und aus einem Hot-Wire Level. 
Beide Szenarien wurden anhand Text und einer Bedienungsanleitung des VR-Controllers ausgiebig erklärt. 
Sobald das WarmUp-Szenario abgeschlossen wurde, folgte der Hauptteil. 
Dieser Bestand aus dem Glücks-Szenario, Langeweile-Szenario und dem Frustrations-Szenario. 
Vor jedem Szenario wurde erneut erklärt, was den Probanden bevorsteht und eine Bedienungsanleitung des VR-Controllers angezeigt. 
Nach jedem Szenario wurde ein Fragebogen angezeigt, welcher in Kapitel 15.2 erklärt wird. 
Nach dem alle Szenarien durchlaufen wurden, kam vor der Danksagung ein Fragebogen über die Geistliche Verfassung. 
In diesem wurde abgefragt, ob der Proband unter psychischen Störung leidet oder sich in einer speziellen medizinischen Behandlung befindet. 
Es wurde keine genaue Angabe gefordert, lediglich ein ja oder ein nein konnten hier angekreuzt werden. 

