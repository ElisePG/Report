\todo[inline]{Verantwortlich: Kevin, Jonas}

Die Hardware besteht im wesentlichen aus zwei Komponenten. Zum einen der Messeinrichtung, mit den einzelnen Sensoren, welche am Kopf einer Probanden befestigt werden werden. Und zum anderen aus dem Messboard, auf dem die empfangenen Signale verarbeitet, und dann weiter gesendet werden. Vom Messboard aus sollten dann alle Daten an einen externen PC übertragen werden, wo diese dann entweder abgespeichert werden, oder aber direkt ausgewertet werden können, um die Gefühlslage des Probanden zu erkennen.

Ziel der Hardware Entwicklung war es letztendlich, ein System zu entwerfen, bei dem die gemessenen Daten möglichst genau (ohne Rauschen) und in Echtzeit (ohne Datenverlust) aufgenommen werden.

In den nachfolgenden Abschnitten werden zunächst die Einwickelten und auch tatsächlich Produzierten PCB‘s (engl. Printed Circuit Board) näher beschrieben. Und welche Denkprozesse und Erwägungen zu diesen Designs geführt haben.

Des weiteren sollen die zur Verwendung gekommen Sensorsysteme, und deren Veränderungen im Laufe der Zeit, erklärt werden. Ebenso soll die Entwicklung der Trägersysteme dieser Sensoren Erwähnung finden. Diese Trägersysteme sollten zum einen eine einfache Messungen ermöglichen, im Idealfall kann der Proband diese ohne Hilfe oder weitere Erklärungen einfach selber aufsetzen. Zum anderen soll natürlich auch auf einen gewissen Tragekomfort geachtet werden. Zu Beginn des Projektes wurde dafür noch ein einfaches Kopfband verwendet, in dem einige Sensoren bereits integriert waren. Gegen Ende des Projektes wurde eine flexible Maske entwickelt, in der alle Sensorsysteme integriert waren, und die relativ einfach aufgesetzt werden konnte.

% Unterkapitel 
%\subsection{Anforderungen} \label{anfoderungen-1}

Auf Grundlage der Ziele des Forschungsprojektes ELISE und dem Sichten und Vergleichen von mehr als 30 wissenschaftlichen Veröffentlichungen der letzten 15 Jahre, ergeben sich bestimmte
Anforderungen für den Entwurf eines eigenen Emotionserkennungssystems. Einige wissenschaftliche Veröffentlichungen sind dabei nicht außer Acht zu lassen. Die Foscher von T.
Sharma, S. Bhardwaj und H. B. Maringanti haben in ihrer Veröffentlichung Emotion Estimation
using Physiological Signals versucht, mit Hilfe von GSR, Herzschlagrate,
BVP und der Temperatur Aufschluss über die Emotionen Zorn, Angst, Freude und Traurigkeit
durch Stimulation verschiedener Songs und Videos zu erhalten. Sie erforschten, in
welchen Fällen sich die Körperleitfähigkeit je nach emotionalem Ausdruck unterschiedlich
verhält. H. F. Garcia, A. A. Orozco und M. A. Alvarez versuchten in ihrer Arbeit Dynamic
physiological signal analysis based on Fisher kernels for emotion recognition durch unterschiedliche Klassifizierungsmodelle, die Signale von EEG, EOG, EMG (Elektromyografie),
GSR, Atmung und Temperatur zu analysieren. Dafür wurden 32 Probanden,
die ein 40-minütiges Video mit Musikausschnitten ansahen, aufgezeichnet und ausgewertet.
Durch ein automatisches Regressionsprozess-Modell verbesserten sie dynamische Merkmale
und weitere aufgezeichnete Signale für eine weiterführende Auswertung.
Die Emotionserkennung erfolgte bis vor einigen Jahren in Verbindung mit zusätzlichen Kameras
und Software zur Gesichtsmimik-Erkennung oder Stimmerkennung, nicht jedoch mit
einer reinen Aufnahme von Körpermesswerten. In ELISE sollen insbesondere die lernrelevanten Emotionen Langweile, Frustration, Verwirrung sowie Engagement und Freude erkannt
werden. Die Hardware-Architektur muss auch in diesem Fall wieder das Ziel erfüllen, dass
das Gefühl der Immersion nicht gestört wird. Das heißt, dass das System zur Erkennung von
lernrelevanten Emotionen an möglichst wenigen Stellen am Körper mit zusätzlicher Sensorik
angebracht wird.
Auf Basis der Literaturrecherche, wie in der Bibliographie ausgewiesen, sind folgende Sensoren
zur Aufnahme der lernrelevanten Emotionen ausgewählt worden:

• Gehirnaktivität (EEG)
• Augenbewegung (EOG)
• Blutvolumenpuls (BVP)
• Sauerstoffsättigung im Blut (PPG)
• Hautleitfähigkeit (GSR)
• Körpertemperatur

Da die Sensorwerte zum Mikrocontroller aufgrund ihres räumlichen Abstandes über den
Bus übertragen werden, unterliegen diese Werte den Zeitanforderungen des Datenbusses.
Hier ist zu überprüfen, welches Buskonzept den Zeitanforderungen gewachsen ist.
Um den Aufwand für den Benutzer gering zu halten und die Immersion nicht zu stören,
wird versucht, die Sensorik direkt an der VR-Brille anzubringen. Für die EEG- und
EOG-Sensoren ist dies sowieso notwendig, da diese Messungen lediglich am Kopf stattfinden
können. Zudem soll das Endsystem echtzeitfähig sein, um in der späteren Anwendung
Änderungen und Fluktuationen der Emotionen erkennen zu können und die Schulungen auf
den Lernenden anzupassen. Das Emotionserkennungssystem soll mobil anwendbar sein, da
neuere Versionen der HTC Vive VR-Brille in Zukunft den kabellosen Betrieb unterstützen.
Auch aus diesem Grund ist die Kompaktheit, Energieeffizienz und die Datenübertragung der
einzelnen Sensoren und die Datenübertragung des späteren Gesamtsystems, die ebenfalls
kabellos stattfinden soll, von großem Interesse. Eine mögliche Stelle zur Unterbringung des
Gesamtsystems wäre am Hinterkopf des Probanden, da dort der nötige Platz vorhanden ist
und erforderliche Befestigungsstellen am Kopfband der HTC Vive von Vorteil sind.



% Unterkapitel 
%\subsection{Konzept} \label{konzept-1}

Auf Abbildung 1 kann man das Konzept der Architektur erkennen, Dies ist zur besseren Anschauung stark vereinfacht. Hierbei bildet der Mikrocontroller das zentrale Element, welches die einzelnen Sensoren anbindet, steuert und die Messsignale grob zur besseren Auswertung verarbeitet. Zur besseren und möglichst in Echtzeit stattfindeten Verarbeitung werden die Daten an einen externen Rechner weitergeleitet. In der Abbildung findet diese Weiterleitung Drahtlos mittels Bluetooth statt. Es wurden in den verschiedenen Prototypen für diese Zwecke sowohl Bluetooth als auch WLAN verwendet. Für die Teilsysteme EEG und EOG wurden schon einfache Physische Filter auf den Leiterplatten vorgesehen, welche die analogen Signale vorberarbeiten, bevor dies von einem AD-Wandler digitalisiert werden. Da die Elektroden für die EEG und EOG Messung nur am Kopf angebracht werden können, empfiehlt es sich die übrigen festgelegten Werte ebenfalls am Kopf zu messen. Um die Messung für möglichst viele Personen mit unterschiedlichen Kopfformen zu ermöglichen wurde zuerst ein elastisches Kopfband und später eine flexible Maske verwendet. All dies wurde für eine spätere Verwendung mit (unter) einer VR-Brille designet. 





% Unterkapitel
\subsection{Hardwareauswahl} \label{hardwareauswahl-1}

Bei der Wahl der richtigen Hardware zur Aufnahme, Verarbeitung und Weiterleitung von
biomedizinischen Signalen, sind die reinen Hardwarekosten von untergeordneter Bedeutung.
Jedoch sollte das Budget für das spätere Gesamtsystem einen gewissen Rahmen nicht überschreiten,
um auch die aufkommenden Endkosten in Verbindung mit einer VR-Brille und der
benötigten Hardware zur Darstellung der Lerninhalte in einem gewissen Rahmen zu halten.
Die Bandbreite der angebotenen Systeme von unterschiedlichen Mikrocontrollern ist dabei
sehr groß. Zu Beginn einer geeigneten Neubeschaffung sollten, wie bei jedem IT-Projekt, die
zuvor genannten Anforderungen an das System betrachtet werden und daraus Auswahlkriterien
für die geeignete Hardware gewählt werden.

% Unterkapitel 
\subsubsection{Auswahlkriterien} \label{auswahlkriterien-1}

Grundvoraussetzung für alle verwendeten Bauteile ist eine Reibungslose Zusammenarbeit mit allen anderen Komponenten. Für den zu bestimmenden Mikrocontroller ergibt sich sich dabei die Anforderung nach möglichst vielen konfigurierbaren Input/Output- Pins (Ios), die möglichkeit zur Drahtlosen Kommunikation (integriertes Bluetooth bzw. WLAN), einen möglichst großen Speicher für Zwischenwerte, ausreichend RAM zur Verarbeitung, sowie mehrere Analog-Digital-Wandler für die Anbindung der analogen Sensoren. Neu entwickelte ARM-basierte CPUs von unterschiedlichen Herstellern besitzen bereits viele der benötigten Schnittstellen, wie zum Beispiel Grafik, Ethernet, CAN, I2C, ADCs, SPI sowie frei konfigurierbare digitale IOs, die bereits in der CPU integriert sind. Durch die Vielseitigkeit der integrierten Schnittstellen, sind die meisten Hardwareanforderungen ohne großen Zusatzaufwand umsetzbar. Ein umfassender Hard- und Softwaresupport verspricht  einen schnellen Einstieg als auch eine effiziente und kostengünstige Umsetzung zur Realisierung
des Gesamtprojekts. Durch die gute Applikationsunterstützung der CPU-Hersteller für die Medizintechnik werden immer mehr Geräte auf Basis der ARM-Architektur entwickelt und genutzt. Auf Basis der zuvor genannten Auswahlkriterien kann nun ein geeigneter Mikrocontroller für das spätere System festgelegt werden.




% Unterkapitel 
\subsubsection{Festlegung der genutzten Hardware} \label{festlegung-subsubsec}

Für den zweiten Prototypen wurde der ursprüngliche Entwurf wieder komplett verworfen. Die wesentliche Neuerung im Vergleich zum vorherigen Prototypen bestand darin, dass die Analogen Signale EEG und EOG nicht mehr wie zuvor mit AD-Wandler und Operationsverstärker bestimmt wurden, sondern mit einem speziell dafür vorgesehenen Chip. Dem ADS1299 der Firma Texas-Instruments. Dieser Chip wurde für die Umwandlung von EEG-Signalen entwickelt, und ist somit in der Lage kleinste Spannungsschwankungen zu erkennen. Dazu hat jeder Kanal des integrierten AD-Wandlers zwei Eingänge, über die an einem Differenzverstärker der Spannungsunterschied zwischen zwei Elektroden gemessen wird. Dies ermöglicht eine Differenzmessung und auch eine Referenzmessung für die EEG-Signale. Dadurch wurde eine massive Platzersparnis im Vergleich zum ersten Prototypen erreicht, da die Verwendung von Operationsverstärkern komplett entfiel. Zudem wurde hier eine höher Auflösung erreicht, was allerdings die Vergrößerung der Datenpakte zur folge hatte, von 8-Bit im ersten Prototypen zu jetzt 24-Bit. Die Daten werden hier über eine SPI-Schnittstelle übertragen. Dadurch wurde ebenfalls ein Wechsel des Verwendeten Mikrocontrollers bedingt. Es galten hier die gleichen Kriterien wie auch zuvor. Letztendlich fiel die Wahl auf den Mikrocontroller ESP32 der Firma Espressif. Dieser verfügt über ausreichend Schnittstellen für I2C,SPI und UART. Außerdem ist hier auch ein ausreichend guter AD-Wandler für das analoge GSR-Signal vorhanden. Wegen der bereits integrierten Antenne wurde hier das Modell ESP-WROOM-32 gewählt. Zusätzlich zu Antenne ist die gesamte benötigte Peripherie ebenfalls schon enthalten, so musste nur ein Spannungspin an 3,3 Volt, mit einem Kondensator zur Spannungsglättung, angeschlossen werden. Dadurch wurde eine weiter Ersparnis in Hinsicht auf den Platzbedarf erzielt. Als letzte Verbesserung wurde hier noch ein Chip vom Typ FT232RL eingefügt, mit der eine Reibungslose Kommunikation, und dadurch Programmierung, zwischen der UART-Schnittstelle der Mikrocontrollers und eine USB-Schnittstelle ermöglicht wurde.








% Unterkapitel
\subsection{Hardwarearchitektur} \label{hardwarearchitektur-subsec}



% Unterkapitel
\subsubsection{GSR-Sensor} \label{gsr-1}

GSR ist eine Abkürzung für das englische Galvanic Skin Response und ist synonym mit der Abkürzung EDA (engl. Elektrodermal activity dt. Elektrodermale Aktivität). Es wird also mit dem GSR-Sensor die Hautleitfähigkeit zwischen zwei leitenden Elektroden gemessen. Für unsere Zwecke, also der Emotionserkennung, ist dies insofern relevant, da einige Emotionen durchaus Einfluss auf die Hautleitfähigkeit eines Probanden nehmen können, so wie zum Beispiel negative Emotionen wie Angst oder Stress die Schweißproduktion des Körpers beeinträchtigen können. Dadurch ändert sich dann natürlich auch die Leitfähigkeit der Haut.

Im wesentlichen wurde die Hautleitfähigkeit mittels zweier verschiedener Messeinrichtungen bestimmt. Bei den ersten beiden Prototypen war der GSR-Sensor im wesentlichen ein Spannungsteiler. Der genaue Aufbau kann der Abbildung X entnommen werden. 

%BILD

Um ein direktes einspeisen der Versorgungsspannung VCC auf den Probanden zu vermeiden wurde der Widerstand R6 (68kOhm) als Strombegrenzung eingeführt. 1 und 2 an JP4 sind die am Probanden befindlichen Elektroden, zwischen denen der Widerstand der Haut (R2) gemessen werden soll, P02 ist die zugehörige Spannung, und wird an den Analog-Digital-Wandler der Mikrocontrollers weitergegeben. C3 ist zum Abfangen von Spannungsspitzen da. Daraus ergibt sich folgende Gleichung für P02:

P02 = Vcc + R1/(R1 + R2)  %Formatieren



% Unterkapitel
\subsubsection{Temperatur-Senosr} \label{temp-1}








% Unterkapitel 
\subsubsection{Pulsoximeter} \label{pulsoximeter-1}

Die Auswahl des Verwendeten Pulsoximeters hatte sich im Laufe des Projektes nicht geändert, eine genaue Beschreibung ist dem Kaptiel 5.4.3 Pulsoximeter zu entnehmen.



% Unterkapitel 
\subsubsection{EEG} \label{eeg-subsubsec}

Die EEG-Schaltung hat sich im vergleich zum zweiten Prototypen nicht mehr Grundlegend geändert, es wurde lediglich eine Minimierung der Schaltung erreicht. eine genauere Beschreibung befindet sich in Kaptiel 11.2.4 EEG.





% Unterkapitel 
\subsubsection{EOG} \label{eog-1}








% Unterkapitel
\subsubsection{Datenübertragung} \label{datenuebertragung-subsubsec}










% Unterkapitel 
\subsection{Programmierung} \label{programmierung-subsec}

Die Programmierung des dritten Prototypen basiert im wesentlichen auf der des zweiten Protoypen, da sich in Bezug auf die Hardware nur die GSR-Schaltung geändert hat. Wie auch bei vorherigen Prototypen erfolgte die Programmierung mit Hilfe der Arduino Plattform und der durch die Hersteller schon zur Verfügung gestellten Bibliotheken.





% Unterkapitel 
\subsection{Aufnahme der {\"u}bertragenen Daten} \label{aufnahme-daten-subsec}

Beim dritten Prototypen wurden die Daten wieder drahtlos übertragen. Im Gegensatz zum ersten Prototypen kam hier aber nicht mehr ein Bluetooth-Modul zum Einsatz, sondern das im ESP32 integrierte WLAN-Modul. In den mit diesem Prototypen durchgeführten Messungen wurden die Daten mittels WireShark aufgenommen und abgespeichert. Diese Daten wurden später zur Datenanalyse in eine CSV-Datei umgewandelt. Zur Übertragung wurde das zuvor schon beschriebene UDP-Protokoll  verwendet. An dieser Stelle sei noch einmal Erwähnt, das alle Daten mit einer Abtastrate von 250 Samples pro Sekunde gemessen wurden. Die einzige Ausnahme hier ist die Temperatur, die lediglich mit einer Rate von 3,3 Samples pro Sekunde gemessen wurde. 

