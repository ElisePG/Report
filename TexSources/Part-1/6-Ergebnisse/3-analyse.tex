\subsection{Analyse der Ergebnisse} \label{analyse-subsec}


Es sei darauf hingewiesen, dass Lösungen, die auf der Verwendung von Deep Neural Networks (DNNs) zur Extraktion von Merkmalen basieren (z.B. Multi-Layer-Perceptron, Convolutional Neural oder Long-Short-Term-Memory Networks) ebenfalls getestet wurden. Diese Ansätze konnten aber nicht so gute Ergebnisse erzielen, wie die handgefertigten Merkmale. Erkennungsraten für die am wenigsten vertretenen Klassen (insbesondere ``Frustration'') scheinen der Grund für die schwache Performance der DNNs zu sein. Wir gehen davon aus, dass dieses Phänomen durch die relativ geringe Größe unseres Datensatzes verursacht wird. \\

Die schächere Performance des CA hat uns auch überrascht. Der Grund hierfür ist aber sehr wahrscheinlich der selbe wie bei den  DNNs, und zwar der relativ kleine Datensatz. \\

Allgemein deuten die Ergebnisse aber darauf hin, dass unser biomedizinisches Datenerfassungssystem zur Emotionserkennung erfolgreich eingesetzt werden könnte, um ein intelligent adaptives Lernsystems zu verbessern. Zukünftige Arbeiten werden die Verfeinerung des Multisensor-Datenerfassungsgerätes, die Erfassung weiterer und größerer Datensätze für die weitere Mustererkennungsanalyse und die Analyse der Wirksamkeit des Emotionserkennungssystems in einem VR-affektiven Lernkontext beinhalten.