\subsection{Ergebnisse des Codebook Approach} \label{ergebnisse-codebook-approach-subsec}


Ähnlich wie bei den handgefertigten Merkmalen haben wir die Daten mit einer Normalisierungstechnik vorverarbeitet und dann die Segmentierung verwendet. Die Zeitfensterparameter sind identlisch wie bei der Studie mit den handgefertigten Merkmalen.
Für die Klassifizierung haben wir SVM mit Soft-Margins und dem RBF-Kernel benutzt. 
Um die optimalen Parameter des SVM-Klassifikators zu bestimmen (z.B. Soft-Margin $C$ und Kernelparameter $\gamma$), wurde hier wieder die Gitter-Suche angewendet. \\


Die Ergebnisse, die wir mit dem CA mit fester Zuordnung (hard assignment) und $C = 8$, $\gamma = 0,002$ für jeden Probanden erhielten, waren 52\%, 38\% und 38\%, was einem Durchschnitt von 42,67\% entspricht. CA mit Soft-Assignment wurden ebenfalls getestet, lieferte aber schlechtere Ergebnisse als CA mit Hard-Assignment. In diesem speziellen Datensatz schneidet der CA also schlechter ab als die handgefertigten Merkmale.