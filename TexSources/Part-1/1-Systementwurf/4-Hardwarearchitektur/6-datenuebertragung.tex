\subsubsection{Daten{\"u}bertragung} \label{datenuebertragung-1}

Da durch externe Rechner leistungsfähigere Möglichkeiten für die Auswertung und Analyse der Daten zur Verfügung stehen sollen diese, ohne größere Bearbeitung im Mikrocontroller selbst, übertragen werden.
Die geschieht insbesondere unter der Betrachtung, dass das entwickelte System später auch zur Echtzeiterkennung von Emotionen verwendet werden soll. 
Da hier im Endeffekt recht anspruchsvolle Algorithmen der Mustererkennung zum Einsatz kommen sollen, macht es Sinn hier schon eine Kommunikation mit einem Leistungsfähigeren System vorzusehen.
Die gemessenen Werte werden in 8-Bit-Datenpaketen Übertragen. Die Übertragung erfolgt vom Mikrocontroller nrf52832 über dessen UART-Schnittstelle an ein externes Bluetooth-Modul vom Typ RN42XVP. 
Für die Übertragung per UART werden lediglich 2 Datenschnittstellen verwendet, die RXD- und TXD-Leitungen.
Zudem benötigt das Bluetooth-Modul nur noch einen Anschluss für die Masseleitung, sowie eine 3,3 Volt Spannungsversorgung. 
Zunächst werden die Daten für die Temperatur und dann alle ADC-Werte für GSR,EEG und EOG (1. Kanal rechts/links, 2, Kanal: oben/unten) übertragen.  
Die nächsten beiden Datenwerte sind die Rohdaten der roten und infraroten LED, welche vom Photosensor des Pulsoximeters gemessen werden. 
Die letzten beiden Daten wurden im Mikrocontroller aus den Rohdaten des Pulsoximeters berechnet. Es handelt sich hierbei um den Puls, sowie die Sauerstoffsättigung des Probanden.

