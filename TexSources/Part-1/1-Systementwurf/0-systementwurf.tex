

Die Hardware besteht im wesentlichen aus zwei Komponenten. Zum einen der Messeinrichtung, mit den einzelnen Sensoren, welche am Kopf einer Probanden befestigt werden werden. Und zum anderen aus dem Messboard, auf dem die empfangenen Signale verarbeitet, und dann weiter gesendet werden. Vom Messboard aus sollten dann alle Daten an einen externen PC übertragen werden, wo diese dann entweder abgespeichert werden, oder aber direkt ausgewertet werden können, um die Gefühlslage des Probanden zu erkennen.
Ziel der Hardware Entwicklung war es letztendlich, ein System zu entwerfen, bei dem die gemessenen Daten möglichst genau (ohne Rauschen) und in Echtzeit (ohne Datenverlust) aufgenommen werden.
In den nachfolgenden Abschnitten werden zunächst die Einwickelten und auch tatsächlich Produzierten PCB‘s (engl. Printed Circuit Board) näher beschrieben. Und welche Denkprozesse und Erwägungen zu diesen Designs geführt haben.
Des weiteren sollen die zur Verwendung gekommen Sensorsysteme, und deren Veränderungen im Laufe der Zeit, erklärt werden. Ebenso soll die Entwicklung der Trägersysteme dieser Sensoren Erwähnung finden. Diese Trägersysteme sollten zum einen eine einfache Messungen ermöglichen, im Idealfall kann der Proband diese ohne Hilfe oder weitere Erklärungen einfach selber aufsetzen. Zum anderen soll natürlich auch auf einen gewissen Tragekomfort geachtet werden. Zu Beginn des Projektes wurde dafür noch ein einfaches Kopfband verwendet, in dem einige Sensoren bereits integriert waren. Gegen Ende des Projektes wurde eine flexible Maske entwickelt, in der alle Sensorsysteme integriert waren, und die relativ einfach aufgesetzt werden konnte.

% Unterkapitel 
\subsection{Anforderungen} \label{anfoderungen-1}

Auf Grundlage der Ziele des Forschungsprojektes ELISE und dem Sichten und Vergleichen von mehr als 30 wissenschaftlichen Veröffentlichungen der letzten 15 Jahre, ergeben sich bestimmte
Anforderungen für den Entwurf eines eigenen Emotionserkennungssystems. Einige wissenschaftliche Veröffentlichungen sind dabei nicht außer Acht zu lassen. Die Foscher von T.
Sharma, S. Bhardwaj und H. B. Maringanti haben in ihrer Veröffentlichung Emotion Estimation
using Physiological Signals versucht, mit Hilfe von GSR, Herzschlagrate,
BVP und der Temperatur Aufschluss über die Emotionen Zorn, Angst, Freude und Traurigkeit
durch Stimulation verschiedener Songs und Videos zu erhalten. Sie erforschten, in
welchen Fällen sich die Körperleitfähigkeit je nach emotionalem Ausdruck unterschiedlich
verhält. H. F. Garcia, A. A. Orozco und M. A. Alvarez versuchten in ihrer Arbeit Dynamic
physiological signal analysis based on Fisher kernels for emotion recognition durch unterschiedliche Klassifizierungsmodelle, die Signale von EEG, EOG, EMG (Elektromyografie),
GSR, Atmung und Temperatur zu analysieren. Dafür wurden 32 Probanden,
die ein 40-minütiges Video mit Musikausschnitten ansahen, aufgezeichnet und ausgewertet.
Durch ein automatisches Regressionsprozess-Modell verbesserten sie dynamische Merkmale
und weitere aufgezeichnete Signale für eine weiterführende Auswertung.
Die Emotionserkennung erfolgte bis vor einigen Jahren in Verbindung mit zusätzlichen Kameras
und Software zur Gesichtsmimik-Erkennung oder Stimmerkennung, nicht jedoch mit
einer reinen Aufnahme von Körpermesswerten. In ELISE sollen insbesondere die lernrelevanten Emotionen Langweile, Frustration, Verwirrung sowie Engagement und Freude erkannt
werden. Die Hardware-Architektur muss auch in diesem Fall wieder das Ziel erfüllen, dass
das Gefühl der Immersion nicht gestört wird. Das heißt, dass das System zur Erkennung von
lernrelevanten Emotionen an möglichst wenigen Stellen am Körper mit zusätzlicher Sensorik
angebracht wird.
Auf Basis der Literaturrecherche, wie in der Bibliographie ausgewiesen, sind folgende Sensoren
zur Aufnahme der lernrelevanten Emotionen ausgewählt worden:

• Gehirnaktivität (EEG)
• Augenbewegung (EOG)
• Blutvolumenpuls (BVP)
• Sauerstoffsättigung im Blut (PPG)
• Hautleitfähigkeit (GSR)
• Körpertemperatur

Da die Sensorwerte zum Mikrocontroller aufgrund ihres räumlichen Abstandes über den
Bus übertragen werden, unterliegen diese Werte den Zeitanforderungen des Datenbusses.
Hier ist zu überprüfen, welches Buskonzept den Zeitanforderungen gewachsen ist.
Um den Aufwand für den Benutzer gering zu halten und die Immersion nicht zu stören,
wird versucht, die Sensorik direkt an der VR-Brille anzubringen. Für die EEG- und
EOG-Sensoren ist dies sowieso notwendig, da diese Messungen lediglich am Kopf stattfinden
können. Zudem soll das Endsystem echtzeitfähig sein, um in der späteren Anwendung
Änderungen und Fluktuationen der Emotionen erkennen zu können und die Schulungen auf
den Lernenden anzupassen. Das Emotionserkennungssystem soll mobil anwendbar sein, da
neuere Versionen der HTC Vive VR-Brille in Zukunft den kabellosen Betrieb unterstützen.
Auch aus diesem Grund ist die Kompaktheit, Energieeffizienz und die Datenübertragung der
einzelnen Sensoren und die Datenübertragung des späteren Gesamtsystems, die ebenfalls
kabellos stattfinden soll, von großem Interesse. Eine mögliche Stelle zur Unterbringung des
Gesamtsystems wäre am Hinterkopf des Probanden, da dort der nötige Platz vorhanden ist
und erforderliche Befestigungsstellen am Kopfband der HTC Vive von Vorteil sind.



% Unterkapitel 
\subsection{Konzept} \label{konzept-1}

Auf Abbildung 1 kann man das Konzept der Architektur erkennen, Dies ist zur besseren Anschauung stark vereinfacht. Hierbei bildet der Mikrocontroller das zentrale Element, welches die einzelnen Sensoren anbindet, steuert und die Messsignale grob zur besseren Auswertung verarbeitet. Zur besseren und möglichst in Echtzeit stattfindeten Verarbeitung werden die Daten an einen externen Rechner weitergeleitet. In der Abbildung findet diese Weiterleitung Drahtlos mittels Bluetooth statt. Es wurden in den verschiedenen Prototypen für diese Zwecke sowohl Bluetooth als auch WLAN verwendet. Für die Teilsysteme EEG und EOG wurden schon einfache Physische Filter auf den Leiterplatten vorgesehen, welche die analogen Signale vorberarbeiten, bevor dies von einem AD-Wandler digitalisiert werden. Da die Elektroden für die EEG und EOG Messung nur am Kopf angebracht werden können, empfiehlt es sich die übrigen festgelegten Werte ebenfalls am Kopf zu messen. Um die Messung für möglichst viele Personen mit unterschiedlichen Kopfformen zu ermöglichen wurde zuerst ein elastisches Kopfband und später eine flexible Maske verwendet. All dies wurde für eine spätere Verwendung mit (unter) einer VR-Brille designet. 





% Unterkapitel
\subsection{Hardwareauswahl} \label{hardwareauswahl-1}



% Unterkapitel 
\subsubsection{Auswahlkriterien} \label{auswahlkriterien-subsubsec}






% Unterkapitel 
\subsubsection{Festlegung der genutzten Hardware} \label{festlegung-subsubsec}

Beim dritten Prototypen haben keine größeren Änderungen an der Hardware-Auswahl stattgefunden. Hauptziel bei diesem Prototypen war es eine weitere Minimierung, und eine dadurch bedingte Verringerung an Störeinflüssen, zu erzielen. Nennenswert ist hier allerdings, dass das GSR-Signal nun ebenfalls über den ADS1299 gemessen wird. Zudem wurden alle externen Anschlüsse über einen SUBD-25 Stecker gebündelt, was das spätere Anschließen an eine Maske erheblich vereinfacht. Zusätzlich ist nur noch eine externe USB-Schnittstelle, zur erstmaligen Programmierung und der späteren Stromversorgung über eine Powerbank, enthalten.








% Unterkapitel
\subsection{Hardwarearchitektur} \label{hardwarearchitektur-1}

In den nächsten Abschnitten werden die verwendeten Sensoren, sowie die dazugehörigen Messschaltungen näher erläutert. Im Rahmen dieser Projektgruppe wurden insgesamt drei Prototypen gefertigt. In einigen Fällen ist der Sensor, und die dazugehörige Schaltung  unverändert geblieben. Bei anderen Sensoren gab es Änderungen. In diesem Fall werden die Sensoren und Schaltungen für jeden Prototypen aufgeführt und erläutert.

Die Auswahl der zur Emotionsbestimmung nötigen Vitaldaten wurde im Rahmen der bereits erwähnten aufbauenden Masterarbeit von David Krönert bestimmt. Dies waren die Körpertemperatur, Sauerstoffsättigung im Blut, Blutvolumenpuls, Hautleitfähigkeit, Gehirnaktivität und die Augenbewegungen. Diese Auswahl von relevanten Vitaldaten wurde im leufe der Projektgruppe nicht mehr geändert. Einzig die genaue Art der Messung hat sich mit den unterschiedlichen Varianten des Messboards geändert. Im folgenden soll deshalb noch eine kurze Erklärung der einzelnen Sensoren erfolgen, und wie sich diese für unterschiedliche Prototypen geändert haben.  

% Unterkapitel
\subsubsection{GSR-Sensor} \label{gsr-subsubsec}






% Unterkapitel
\subsubsection{Temperatur-Senosr} \label{temp-1}








% Unterkapitel 
\subsubsection{Pulsoximeter} \label{pulsoximeter-subsubsec}








% Unterkapitel 
\subsubsection{EEG} \label{eeg-subsubsec}

Die EEG-Schaltung hat sich im vergleich zum zweiten Prototypen nicht mehr Grundlegend geändert, es wurde lediglich eine Minimierung der Schaltung erreicht. eine genauere Beschreibung befindet sich in Kaptiel 11.2.4 EEG.





% Unterkapitel 
\subsubsection{EOG} \label{eog-subsubsec}

Wie auch die EEg-Schaltung hat sich die EOG-Schaltung der dritten Protoypen im Verlgeich zum zweiten Prototypen nicht mehr geändert. Auch hier wurde nur eine Minimierung vorgenommen.






% Unterkapitel
\subsubsection{Daten{\"u}bertragung} \label{datenuebertragung-1}

Da durch externe Rechner leistungsfähigere Möglichkeiten für die Auswertung und Analyse der Daten zur Verfügung stehen sollen diese, ohne größere Bearbeitung im Mikrocontroller selbst, übertragen werden.
Die geschieht insbesondere unter der Betrachtung, dass das entwickelte System später auch zur Echtzeiterkennung von Emotionen verwendet werden soll. 
Da hier im Endeffekt recht anspruchsvolle Algorithmen der Mustererkennung zum Einsatz kommen sollen, macht es Sinn hier schon eine Kommunikation mit einem Leistungsfähigeren System vorzusehen.
Die gemessenen Werte werden in 8-Bit-Datenpaketen Übertragen. Die Übertragung erfolgt vom Mikrocontroller nrf52832 über dessen UART-Schnittstelle an ein externes Bluetooth-Modul vom Typ RN42XVP. 
Für die Übertragung per UART werden lediglich 2 Datenschnittstellen verwendet, die RXD- und TXD-Leitungen.
Zudem benötigt das Bluetooth-Modul nur noch einen Anschluss für die Masseleitung, sowie eine 3,3 Volt Spannungsversorgung. 
Zunächst werden die Daten für die Temperatur und dann alle ADC-Werte für GSR,EEG und EOG (1. Kanal rechts/links, 2, Kanal: oben/unten) übertragen.  
Die nächsten beiden Datenwerte sind die Rohdaten der roten und infraroten LED, welche vom Photosensor des Pulsoximeters gemessen werden. 
Die letzten beiden Daten wurden im Mikrocontroller aus den Rohdaten des Pulsoximeters berechnet. Es handelt sich hierbei um den Puls, sowie die Sauerstoffsättigung des Probanden.





% Unterkapitel 
\subsection{Programmierung} \label{programmierung-1}







% Unterkapitel 
\subsection{Aufnahme der {\"u}bertragenen Daten} \label{aufnahme-daten-subsec}

Obwohl mit diesem Prototypen keine Messungen mit Emotionsinduktionen an Probanden durchgeführt wurden, fanden zu Testzwecken natürlich auch hier Datenaufnahmen statt. Diese Daten wurden auch wieder mit SerialPlot erfasst, da hier durch die visuelle Darstellung eine relativ schnelle Beurteilung der Daten möglich war. Anders als im Vorhergehenden Prototypen wurde allerdings nicht mehr auf die Bluetooth-Schnittelle zurückgegriffen. Hier wurden die Daten noch über die eingebaute USB-Schnittstelle übertragen, um mögliche Übertragungsfehler einer drahtlosen Verbindung auszuschließen. Das im ESP32 integrierte WLAN-Modul kam hier also noch nicht zum Einsatz.

