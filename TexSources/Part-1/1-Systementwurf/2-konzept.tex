\subsection{Konzept} \label{konzept-1}

Auf Abbildung 1 kann man das Konzept der Architektur erkennen, Dies ist zur besseren Anschauung stark vereinfacht. Hierbei bildet der Mikrocontroller das zentrale Element, welches die einzelnen Sensoren anbindet, steuert und die Messsignale grob zur besseren Auswertung verarbeitet. Zur besseren und möglichst in Echtzeit stattfindeten Verarbeitung werden die Daten an einen externen Rechner weitergeleitet. In der Abbildung findet diese Weiterleitung Drahtlos mittels Bluetooth statt. Es wurden in den verschiedenen Prototypen für diese Zwecke sowohl Bluetooth als auch WLAN verwendet. Für die Teilsysteme EEG und EOG wurden schon einfache Physische Filter auf den Leiterplatten vorgesehen, welche die analogen Signale vorberarbeiten, bevor dies von einem AD-Wandler digitalisiert werden. Da die Elektroden für die EEG und EOG Messung nur am Kopf angebracht werden können, empfiehlt es sich die übrigen festgelegten Werte ebenfalls am Kopf zu messen. Um die Messung für möglichst viele Personen mit unterschiedlichen Kopfformen zu ermöglichen wurde zuerst ein elastisches Kopfband und später eine flexible Maske verwendet. All dies wurde für eine spätere Verwendung mit (unter) einer VR-Brille designet. 



