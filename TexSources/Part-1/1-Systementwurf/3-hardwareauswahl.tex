\subsection{Hardwareauswahl} \label{hardwareauswahl-1}

Bei der Wahl der richtigen Hardware zur Aufnahme, Verarbeitung und Weiterleitung von
biomedizinischen Signalen, sind die reinen Hardwarekosten von untergeordneter Bedeutung.
Jedoch sollte das Budget für das spätere Gesamtsystem einen gewissen Rahmen nicht überschreiten,
um auch die aufkommenden Endkosten in Verbindung mit einer VR-Brille und der
benötigten Hardware zur Darstellung der Lerninhalte in einem gewissen Rahmen zu halten.
Die Bandbreite der angebotenen Systeme von unterschiedlichen Mikrocontrollern ist dabei
sehr groß. Zu Beginn einer geeigneten Neubeschaffung sollten, wie bei jedem IT-Projekt, die
zuvor genannten Anforderungen an das System betrachtet werden und daraus Auswahlkriterien
für die geeignete Hardware gewählt werden.

% Unterkapitel 
\subsubsection{Auswahlkriterien} \label{auswahlkriterien-subsubsec}






% Unterkapitel 
\subsubsection{Festlegung der genutzten Hardware} \label{festlegung-subsubsec}

Beim dritten Prototypen haben keine größeren Änderungen an der Hardware-Auswahl stattgefunden. Hauptziel bei diesem Prototypen war es eine weitere Minimierung, und eine dadurch bedingte Verringerung an Störeinflüssen, zu erzielen. Nennenswert ist hier allerdings, dass das GSR-Signal nun ebenfalls über den ADS1299 gemessen wird. Zudem wurden alle externen Anschlüsse über einen SUBD-25 Stecker gebündelt, was das spätere Anschließen an eine Maske erheblich vereinfacht. Zusätzlich ist nur noch eine externe USB-Schnittstelle, zur erstmaligen Programmierung und der späteren Stromversorgung über eine Powerbank, enthalten.






