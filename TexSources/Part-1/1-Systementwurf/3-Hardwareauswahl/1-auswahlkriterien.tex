\subsubsection{Auswahlkriterien} \label{auswahlkriterien-1}

Grundvoraussetzung für alle verwendeten Bauteile ist eine Reibungslose Zusammenarbeit mit allen anderen Komponenten. Für den zu bestimmenden Mikrocontroller ergibt sich sich dabei die Anforderung nach möglichst vielen konfigurierbaren Input/Output- Pins (Ios), die möglichkeit zur Drahtlosen Kommunikation (integriertes Bluetooth bzw. WLAN), einen möglichst großen Speicher für Zwischenwerte, ausreichend RAM zur Verarbeitung, sowie mehrere Analog-Digital-Wandler für die Anbindung der analogen Sensoren. Neu entwickelte ARM-basierte CPUs von unterschiedlichen Herstellern besitzen bereits viele der benötigten Schnittstellen, wie zum Beispiel Grafik, Ethernet, CAN, I2C, ADCs, SPI sowie frei konfigurierbare digitale IOs, die bereits in der CPU integriert sind. Durch die Vielseitigkeit der integrierten Schnittstellen, sind die meisten Hardwareanforderungen ohne großen Zusatzaufwand umsetzbar. Ein umfassender Hard- und Softwaresupport verspricht  einen schnellen Einstieg als auch eine effiziente und kostengünstige Umsetzung zur Realisierung
des Gesamtprojekts. Durch die gute Applikationsunterstützung der CPU-Hersteller für die Medizintechnik werden immer mehr Geräte auf Basis der ARM-Architektur entwickelt und genutzt. Auf Basis der zuvor genannten Auswahlkriterien kann nun ein geeigneter Mikrocontroller für das spätere System festgelegt werden.


