\subsubsection{Festlegung der genutzten Hardware} \label{festlegung-1}

Für den ersten Prototypen wurde, nach Vergleichen einer Vielzahl auf dem Markt erhältlichen Mikrocontroller unter Berücksichtigung der zuvor bereits erwähnten Kriterien, für das ELISE-Projekt letztendlich der Mikrocontroller nrf52832 ausgewählt. Dieser Mikrocontroller verfügt über alle für das Projekt benötigten Schnittstellen, wie I2C,SPI,UART sowie ausreichende AD-Wandler. Zudem besitzt der nrf52832 einen integrierten 2,4 GHz Funk, mit dem eine drahtlose Kommunikation mit externen Geräten zur Datenauswertung möglich ist. Für unser Projekt wurde das Breakout-Board der Firma Sparkfun gewählt. Dort waren alle benötigten Funktionen sowie die zum Betrieb nötige Peripherie bereits vorhanden, was eine schnelle und unkomplizierte Verwendung ermöglichte.




