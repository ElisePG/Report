\subsection{Hardwarearchitektur} \label{hardwarearchitektur-1}

In den nächsten Abschnitten werden die verwendeten Sensoren, sowie die dazugehörigen Messschaltungen näher erläutert. Im Rahmen dieser Projektgruppe wurden insgesamt drei Prototypen gefertigt. In einigen Fällen ist der Sensor, und die dazugehörige Schaltung  unverändert geblieben. Bei anderen Sensoren gab es Änderungen. In diesem Fall werden die Sensoren und Schaltungen für jeden Prototypen aufgeführt und erläutert.

Die Auswahl der zur Emotionsbestimmung nötigen Vitaldaten wurde im Rahmen der bereits erwähnten aufbauenden Masterarbeit von David Krönert bestimmt. Dies waren die Körpertemperatur, Sauerstoffsättigung im Blut, Blutvolumenpuls, Hautleitfähigkeit, Gehirnaktivität und die Augenbewegungen. Diese Auswahl von relevanten Vitaldaten wurde im leufe der Projektgruppe nicht mehr geändert. Einzig die genaue Art der Messung hat sich mit den unterschiedlichen Varianten des Messboards geändert. Im folgenden soll deshalb noch eine kurze Erklärung der einzelnen Sensoren erfolgen, und wie sich diese für unterschiedliche Prototypen geändert haben.  

% Unterkapitel
\subsubsection{GSR-Sensor} \label{gsr-subsubsec}






% Unterkapitel
\subsubsection{Temperatur-Senosr} \label{temp-1}








% Unterkapitel 
\subsubsection{Pulsoximeter} \label{pulsoximeter-subsubsec}








% Unterkapitel 
\subsubsection{EEG} \label{eeg-subsubsec}

Die EEG-Schaltung hat sich im vergleich zum zweiten Prototypen nicht mehr Grundlegend geändert, es wurde lediglich eine Minimierung der Schaltung erreicht. eine genauere Beschreibung befindet sich in Kaptiel 11.2.4 EEG.





% Unterkapitel 
\subsubsection{EOG} \label{eog-subsubsec}

Wie auch die EEg-Schaltung hat sich die EOG-Schaltung der dritten Protoypen im Verlgeich zum zweiten Prototypen nicht mehr geändert. Auch hier wurde nur eine Minimierung vorgenommen.






% Unterkapitel
\subsubsection{Daten{\"u}bertragung} \label{datenuebertragung-1}

Da durch externe Rechner leistungsfähigere Möglichkeiten für die Auswertung und Analyse der Daten zur Verfügung stehen sollen diese, ohne größere Bearbeitung im Mikrocontroller selbst, übertragen werden.
Die geschieht insbesondere unter der Betrachtung, dass das entwickelte System später auch zur Echtzeiterkennung von Emotionen verwendet werden soll. 
Da hier im Endeffekt recht anspruchsvolle Algorithmen der Mustererkennung zum Einsatz kommen sollen, macht es Sinn hier schon eine Kommunikation mit einem Leistungsfähigeren System vorzusehen.
Die gemessenen Werte werden in 8-Bit-Datenpaketen Übertragen. Die Übertragung erfolgt vom Mikrocontroller nrf52832 über dessen UART-Schnittstelle an ein externes Bluetooth-Modul vom Typ RN42XVP. 
Für die Übertragung per UART werden lediglich 2 Datenschnittstellen verwendet, die RXD- und TXD-Leitungen.
Zudem benötigt das Bluetooth-Modul nur noch einen Anschluss für die Masseleitung, sowie eine 3,3 Volt Spannungsversorgung. 
Zunächst werden die Daten für die Temperatur und dann alle ADC-Werte für GSR,EEG und EOG (1. Kanal rechts/links, 2, Kanal: oben/unten) übertragen.  
Die nächsten beiden Datenwerte sind die Rohdaten der roten und infraroten LED, welche vom Photosensor des Pulsoximeters gemessen werden. 
Die letzten beiden Daten wurden im Mikrocontroller aus den Rohdaten des Pulsoximeters berechnet. Es handelt sich hierbei um den Puls, sowie die Sauerstoffsättigung des Probanden.



