\subsection{Aufnahme der {\"u}bertragenen Daten} \label{aufnahme-daten-1}

Da die aufgenommenen Daten später mit geeigneten Algorithmen der Mustererkennung eingelesen und genauer analysiert werden sollen, musste zunächst ein geeignetes Datenformat zur Speicherung gefunden werden.
Letztendlich wurden die Daten in einer Comma-seperated Values Datei (kurz CSV) gespeichert. 
Die Aufnahme der Daten erfolgte, nach der Übertragung via Bluetooth an einen geeigneten Rechner,  mit Hilfe der Freeware SerialPlot. Mit dieser Software war es möglich die Daten schon in Echtzeit anzusehen. 
So war es auch möglich die ankommenden Daten auf individuelle Kanäle zu verteilen, und so jeden Wert Separat zu betrachten. Diese Funktion war insbesondere bei der Überprüfung, ob jeder Sensor auch korrekt funktioniert und angeschlossen ist nützlich. 
Für eine nachfolgende Analyse, unabhängig von der Msutererkennung, konnten die in der CSV-Datei gespeicherten Daten auch mittels SerialPlot direkt wieder eingelesen werden. 