\todo[inline]{Verantwortlich: Minas}


Das Kapitel beinhaltet einen Ablauf, die verwendeten Fragebögen und die genutzten Szenarien (Glück, Langeweile, Frustration) der Emotionsinduktion für den ersten Prototypen. 
Kapitel Ablauf beschreibt in welcher Reihenfolge die Szenarien und die Fragebögen aufgerufen wurden. 
In den Kapiteln Fragebogen, Glücks- ,Langeweile- und Frustration-Szenario wird, die Entwicklung und Implementierung erläutert. 
Die Emotioninduktion des ersten Prototypen wurde nicht in einer VR-Umgebung entwickelt. 
Zudem muss angemerkt werden, dass diese Emotionsinduktion für eine Zwischenstudie diente. 
Dadurch wurde der Fokus nicht auf die Implementierung gelegt, welche verschiedene Formate besaß z.B. HTML oder PowerPoint, sondern auf den Inhalt, welches der Proband zu sehen oder zu hören bekam.
Die Emotionsinduktion wurde sowohl in Englisch als auch in Deutsch angeboten.



% Unterkapitel
\subsection{Ablauf} \label{ablauf-subsubsec}


\todo[inline]{Verantwortlich: Minas}


% Unterkapitel 
\subsection{Fragebogen} \label{fragebogen-4}


\todo[inline]{Verantwortlich: Boris}

F{\"u}r dieses Prototyp wurden zwei Arten von Frageb{\"o}gen benutzt. 
Der erste Typ ist sehr {\"a}hnlich mit dem von der zweite Prototyp. Es enth{\"a}lt in dem informativen Teil einen Text, wo es beschrieben wird, wie der Fragebogen ausgef{\"u}llt werden soll. Der andere Teil besteht aus vier Dropdown-Boxen von dreizehn Optionen, die zw{\"o}lf verschiedene Emotionen und einen als null oder neutral geltenden Zustand enthalten. Jede Dropdown-Box entspricht ein Viertelzeit der Szenario. Es soll zwischen die Optionen jeder Dropdown-Box gew{\"a}hlt werden, welche Emotion es am st{\"a}rksten empfindet wird, je nachdem, wann man es f{\"u}hlte, d.h. ob es das erste, zweite, dritte oder letzte Quartal der Zeit des Videos war, um die Emotion zu bew{\"a}ltigen. Unten gibt es ein Button wo man dr{\"u}cken kann, wenn man fertig ist. Allerdings hat man auch die M{\"o}glichkeit seine Wahl zu {\"a}ndern, auch wenn man sich schon im n{\"a}chsten Schritt befindet, indem man in diesem n{\"a}chsten Schritt auf den Zur{\"u}ck-Button dr{\"u}ckt und die gew{\"u}nschten {\"a}nderungen vornimmt. Hierf{\"u}r wird ein Widget Blueprint erstellt, die vier Dropdown-Boxen mit dem gew{\"u}nschten Anzahl an Optionen hinzugef{\"u}gt und die Labels von der unterschiedlichen Optionen der Dropdown-Boxen definiert. Ein Button ``next'' wird auch erstellt um zum n{\"a}chsten Fragebogen zu navigieren. Es wird auch eine zwischen Speicherungsfunktion in ein anderes Skript definiert, die hier aufgerufen wird, um die {\"a}nderung auch nach das dr{\"u}cken von dem ``next'' Button zu Speichern. Dabei wird vier Variablen definiert und die Werte von der gew{\"a}hlten Optionen werden ihnen zugewiesen. \\

(Bild f{\"u}r Fragebogen) \\


Der zweite Typ {\"a}hnelt dem ber{\"u}hmten Modell von James Russels ``circumplex'' \cite{russel_1980}. Es ist ein klassisches Modell mit einer kreisf{\"o}rmigen Struktur, die auf zwei senkrechten Diagonalen ruht. Die vertikale Achse, die die Erregung darstellt, und die horizontale Achse, die die Valenz darstellt. Das Zentrum des Kreises stellt eine neutrale Valenz und ein mittleres Erregungsniveau dar. Andere Emotionen werden auf jeder Ebene des Kreises dargestellt.  Hier wird ein weniger bekanntes Modell verwendet, das ``Self Assessment Manikin'' (SAM). Es besteht aus drei Reihen mit je f{\"u}nf Piktogrammen. Diese Piktogramme stellen den Zustand eines Gesichts nach verschiedenen Arten von Emotionen dar. So repr{\"a}sentiert der erste Bereich die Wertigkeit, der zweite die Erregung und der dritte die Dominanz. Eine Erkl{\"a}rung zu jedem dieser Begriffe ist ebenfalls neben dem Fragebogen enthalten, um die Testpersonen {\"u}ber diese W{\"o}rter aufzukl{\"a}ren.  Bei jeder Avatar und in der Mitte jeder der beiden Avatar befindet sich ein Checkbox.  So muss man f{\"u}r jede Zeile das Checkbox ausw{\"a}hlen, das ihrem emotionalen Zustand am besten entspricht. Man kann nur ein Checkbox pro Zeile markieren und man hat auch die M{\"o}glichkeit wie bei dem ersten Model seine Wahl zu {\"a}ndern. Es kann einfach mit Branch-Bedingungen realisieren werden.  Diese werden auch in eine Widget Blueprint wie f{\"u}r das erste Modell gemacht. Es gibt auch wieder die zwischen Speicherungsfunktion und das Button ``next''. Was neues hier kommt ist das Button ``back'' um wieder zum ersten Fragebogen zu navigieren. 

(Bild f{\"u}r circumplex) \\

% Unterkapitel 
\subsection{Szenarien} \label{szenarien-1}

\todo[inline]{Verantwortlich: Meryem}


Im Folgenden Kapitel werden drei verschiedene Szenarien dargeboten. Die Emotionen Glück, Langweile und Frustration werden hierfür zunächst erläutert. Für jede dieser Emotionen werden Szenarien vorgestellt, bei welchen diese bei den Probanden ausgelöst bzw angeregt werden sollen.


% Unterkapitel 
\subsubsection{Glück} \label{glueck-4}



\todo[inline]{Verantwortlich: Minas}




Um das Glück-Szenario in einer VR-Umgebung zu verwirklichen, wurden sich für HDR-Panorama-Bilder entschieden, die in Unreal eine Umgebung bilden sollen. 
Der Grund weswegen die Bilder HDR und Panorama sein müssen, werden im Laufe dieses Kapitels erklärt. 
Außerdem wird eine Audio-Datei im Hintergrund abgespielt und ein Text eingeblendet, welches thematisch zum Bild passt. 
Die Grundidee stammt vom ersten Prototypen (Kapitel 7.3.1), welches nicht in einer VR-Umgebung gelöst wurde. 
In Kapitel 7.3.1 wurde zudem erklärt, weshalb und welche Audio-Datei im Hintergrund abgespielt wird, weshalb die Texte eingeblendet werden und was sich unter Glück verstehen lässt. \\

Insgesamt besteht das Glück-Szenario aus acht HDR-Panorama-Bilder für das Hauptszenario und ein HDR-Panorama-Bild für das Warm-Up-Szenario. 
Abbildung \ref{fig-glueck4} zeigt alle Bilder die genutzt werden und deren Texte (falls vorhanden). \\

\begin{figure}[H] \centering
\includegraphics[width=15cm]{Images/gluck4.png} 
\vspace{-0.3cm} 
\caption[Im Glück-Szenario verwendete Bilder und deren Texte]{Im Glück-Szenario verwendete Bilder und deren Texte\cite{sun360}.}
\label{fig-glueck4} 
\end{figure}


Die Bilder lassen sich jedoch nicht ohne weiteres in Unreal einbinden. 
Um dies zu realisieren wurde sich für eine CubeMap in DDS-Format entschieden. 
Hierfür müssen die Bilder zunächst durch verschiedene Tools (Blender, Photoshop) bearbeitet werden und dann mit bestimmten Konfigurationen in Unreal eingebunden werden. \\

\begin{figure}[H] \centering
\includegraphics[width=\textwidth]{Images/hdr-panorama.png} 
\caption{Ein im Projekt verwendetes HDR-Panorama-Bild.}
\label{fig-hdr} 
\end{figure}


Anhand Abbildung \ref{fig-hdr} wird die Bearbeitung der Bilder erklärt. 
Nach Auswahl eines Panorma-HDR-Bildes, wird dieses in Blender bearbeitet. 
Blender wird benötigt, um aus dem gesamten Bild, sechs Einzelbilder mit der benötigten Rotation zu erzeugen. 
Das ist der erste Schritt um später eine CubeMap zu erzeugen. 
Im Internet existiert bereits eine Blender-File mit den benötigten Konfigurationen\cite{facerig19}, um an diese Einzelbilder zu gelangen. 
Diese besteht aus einer Kamera, welches um das importierte Bild rotiert und dieses zurecht schneidet. 
Handelt es sich um kein Panorama Bild, werden die sechs Bilder falsch geschnitten und können dadurch die Kriterien einer CubeMap nicht erfüllen. \\

\begin{figure}[H] \centering
\includegraphics[width=\textwidth]{Images/blender-bilder.png} 
\caption{Ausgabe der Blender-File und deren Rotation.}
\label{fig-blender-bilder} 
\end{figure}


Um eine CubeMap in Unreal einzubinden, wird jedoch nur ein Bild im DDS-Format benötigt und nicht sechs Einzelbilder. 
DDS steht für Direct Draw Surface und ist ein entwickeltes Format von Microsoft. 
Dieses Format wird hauptsächlich für die Speicherung von CubeMaps und Texturen verwendet und erhöht die Geschwindigkeiten in Spielen ohne Verlust von Details\cite{dateiendungen19}.
Nvidia bietet ein Plugin für einige Adobe Photoshop Versionen und GIMP, um CubeMaps in DDS-Format abzuspeichern.
Der nächste Schritt eine CubeMap zu erstellen, wurde mit Adobe Photoshop CS2 und dem Nvidia Texture- Plugin realisiert. 
Hierfür müssen die Bilder in Abbildung \ref{fig-blender-bilder} in richtiger Reihenfolge aneinander gereiht werden. 
Dies geschieht in Adobe Photoshop. 
Die benötigte Reihenfolge wird in Abbildung \ref{fig-rotation} gezeigt. \\

\begin{figure}[H] \centering
\includegraphics[width=\textwidth]{Images/rotation.png} 
\caption[Rotation und Reihenfolge für eine CubeMap]{Rotation und Reihenfolge für eine CubeMap\cite{franczak19}.}
\label{fig-rotation} 
\end{figure}


Somit ergibt sich für die Bearbeitung der Abbildung \ref{fig-blender-bilder}, Abbildung \ref{fig-blender-bilder2}. \\

\begin{figure}[H] \centering
\includegraphics[width=\textwidth]{Images/blender-bilder2.png} 
\caption{Rotation und Reihenfolge für die CubeMap der Abbildung \ref{fig-blender-bilder}.}
\label{fig-blender-bilder2} 
\end{figure}


Nach einer aneinander Reihung der einzelnen Bilder, lässt sich das dadurch entstandene Bild mit Hilfe des Nvidia Plugins in eine DDS-File exportieren. 
Somit wird die gewünschte Cubemap generiert.
Abbildung \ref{fig-cubemap} zeigt die aufgeklappte Form einer CubeMap.
Jetzt kann die DDS-File in Unreal importiert bzw. konfiguriert werden. \\

\begin{figure}[H] \centering
\includegraphics[width=\textwidth]{Images/cubemap.png} 
\caption[Aufgeklappte CubeMap]{Aufgeklappte CubeMap\cite{belanec19}.}
\label{fig-cubemap} 
\end{figure}


Sobald die DDS-File im Unreal-Projekt importiert wurde, muss an der DDS-File selbst Einstellungen vorgenommen werden. 
Abbildung \ref{fig-einstellungen} beinhaltet die optimalen Einstellungen. \\

\begin{figure}[H] \centering
\includegraphics[width=\textwidth]{Images/einstellungen.png} 
\caption{Optimale Einstellungen des Bildes in Unreal.}
\label{fig-einstellungen} 
\end{figure}


Diese Einstellungen lassen sich für alle Bilder übernehmen. 
Lediglich die Einstellung RGBCurve unter ``Adjustments'' muss von Bild zu Bild variiert werden. 
Desto höher der eingetragene Wert, umso kräftiger werden die Farben des Bildes. 
Wird der Standardwert von ``1,0'' gelassen, wirkt das Bild sehr blass und ist damit nicht anschaulich. 
Hier spielt auch die Qualität des Bildes eine große Rolle. 
Am Anfang des Kapitels wurde HDR erwähnt. 
HDR steht für High Dynamic Range und ermöglicht ein größeren Helligkeitsbreich als SDR (Standard Dynamic Range). 
Es lässt das Bild realistischer wirken, ohne Farbtöne im dunklen oder hellen Bereichen zu vernachlässigen\cite{eizo19}.
Abbildung \ref{fig-sdr-hdr} zeigt den Qualitätsunterschied zwischen SDR und HDR.
Es ist von Vorteil ein solches Format zu verwenden, um eine gute VR-Umgebung zu realisieren. \\

\begin{figure}[H] \centering
\includegraphics[width=8cm]{Images/sdr.png} 
\caption[SDR vs. HDR]{SDR vs. HDR\cite{finch19}.}
\label{fig-sdr-hdr} 
\end{figure}


Nach dem das Bild in Unreal fertig konfiguriert wurde, muss eine ``Blueprint Class'' angelegt werden. 
Diese wird in diesem Beispiel ``SkySphere-BP'' genannt. 
In diesem Blueprint wird eine Variable angelegt mit dem Namen ``SkyMaterial''. Außerdem sollte das Construction Script wie in Abbildung \ref{fig-skysphere} aufgebaut sein. \\

\begin{figure}[H] \centering
\includegraphics[width=8cm]{Images/skysphere.png} 
\caption{SkySphere-BP Construction Script.}
\label{fig-skysphere} 
\end{figure}


Mit diesem Construction Script lassen sich CubeMaps in die SkySphere importieren und bei Bedarf auch austauschen. \\

Nun muss ein Material für die SkySphere erstellt werden. Abbildung \ref{fig-konfiguration} zeigt wie diese Konfiguriert werden sollte. \\

\begin{figure}[H] \centering
\includegraphics[width=8cm]{Images/konfiguration.png} 
\caption{Konfigurationen von Material.}
\label{fig-konfiguration} 
\end{figure}


Im Knoten ``ParamCube'' muss die Textur ausgesucht werden. 
In diesem Fall handelt es sich um die erstellte CubeMap. 
Als letztes muss das erzeugte Material der SkySphere zugewiesen werden. 
Nun lässt sich die Umgebung in Unreal anzeigen.
Der Text im Bild lässt sich mit einem Blueprint-Pawn erzeugen. 
Die Position des Textes wird manuell vorgenommen, in dem der Text im Bild verschoben wird. 
Damit der Übergang zwischen den Bildern angenehm für die Probanden wirken soll, wird ein ``fade in'' und ein ``fade out'' für jedes einzelne Bild erzeugt. 
Hierfür muss in Unreal unter Cinematics eine Matinee hinzugefügt werden. 
In dieser Matinee wird das Fade erzeugt, in dem die Anzeigedauer angegeben wird und vier Keys hinzugefügt werden. 
Der erste Key sagt dem Fade bei welcher Sekunde das Bild anfängt, der zweite Key bis wann das Bild seine maximale Helligkeit erreichen soll, der dritte Key ab wann das Bild wieder dunkler werden soll und der vierte Key bis wann das Bild komplett verschwunden sein soll. 
Abbildung \ref{fig-matinee} zeigt das Konfigurationsfenster der Matinee. \\

\begin{figure}[H] \centering
\includegraphics[width=\textwidth]{Images/matinee.png} 
\caption[Konfiguration der Matinee]{Konfiguration der Matinee; rote Dreiecke = Keys; grüner Bereich = Anzeigedauer.}
\label{fig-matinee} 
\end{figure}


Somit ist das erste Bild für das Szenario fertig konfiguriert. 
Dieser Ablauf wurde für acht weitere Bilder durchgeführt. \\

Der Wechsel zwischen den Bildern für das Hauptszenario wurde anhand der Level-Option von Unreal vorgenommen. 
Diese Option besteht aus einem anhaltenden Level und beliebig viele Level, die hinzugefügt werden können. 
In diesem Fall acht Level für die verwendeten Bilder. 
Das anhaltende Level beinhaltet die Audio-Datei und verwaltet den Wechsel der Bilder, welches alle 39 Sekunden durchgeführt wird. Die 39 Sekunden kamen zustande, in dem die Länge der Audio-Datei durch die Anzahl der Bilder geteilt wurde. 
Damit wurde vermieden das die Audio-Datei von Anfang an abgespielt wird. 
Ein großer Vorteil der Level-Option ist, dass die Audio-Datei fortläuft, auch während dem Level-Wechsel und somit keine Unterbrechungen entstehen. 
Da das WarmUp für das Glücks-Szenario nur aus einem Bild besteht, musste hier keine Level-Option konfiguriert werden. \\



% Unterkapitel 
\subsubsection{Langeweile} \label{langeweile-1}

\todo[inline]{Verantwortlich: Boris\\}

Ärger wird im Allgemeinen als eine Emotion wahrgenommen, die man erlebt, wenn man einen Zustand durchläuft, an dem man kein Interesse hat \cite{vodanovich_2003}. Dieses Gefühl charakterisiert Inaktivität, Unproduktivität und kann manchmal zu Melancholie oder Traurigkeit führen.
In dieser Studie wird das Erwachen der Langeweile durch eine grüne rotierende Scheibe mit einem schwarzen Strahl verursacht. \\

(Bild für langweile)\\

 Das Prinzip besteht darin, die Scheibe durch die Testperson durch Drücken einer Taste am Joystick alle 15 Sekunden oder etwas länger für 5 Minuten drehen zu lassen. Direkt danach wird die Testperson zu einem neuen Fragebogen geführt, den sie in dieser Zeit im Verhältnis zu dem, was sie bei diesem Test empfunden hat, ausfüllen muss.

% Unterkapitel 
\subsubsection{Frustration} \label{frust-4}

\todo[inline]{Verantwortlich: Meryem}



Wie bereits in Abschnitt 7.3.3 dargeboten, stellt die Frustration einen negativen Zustand des Menschen dar, welcher durch Misserfolgserlebnissesowie durch Versagungs- und Entt{\"a}uschungserlebnisse einhergehen kann. Die Frustration stellt einen negativen Zustand des Menschen dar, welcher mehrere Indikatoren haben kann. Dieser Zustand kann sowohl eine Gef{\"u}hlslage als auch eine Folge vorhergehender Emotionen sein. \\

F{\"u}r das Frustrationsexpermiment der Studie soll erneut auf das Ausl{\"o}sen der Misserfolgserlebnisse sowie eine empfundene Ungerechtigkeit zur{\"u}ckgegriffen werden. \\

Dem Probanden wird die Aufgabe gestellt, in eine VR-Umgebung das Spiel ``hei{\ss}er Draht'' zu spielen. In diesem Spiel besteht die Aufgabe darin, den Ring, welcher sich in der Hand des Spieler befindet, von dem Startpunkt zum Endpunkt eines Drahts zu bef{\"o}rdern. Der Ring darf w{\"a}hrenddessen den Draht nicht ber{\"u}hren.Dieses Spiel ist durch die geschwungene Form des Drahts schwierig und fordert daher viel Ruhe und Geschick. Dem Probanden wird eine Zeit vorgegeben, in welcher dieser von dem Startpunkt zum Endpunkt gelangen muss. Der Faktor des Zeitdrucks kann eine Stressreaktion ausl{\"o}sen. Da das Spiel jedoch den Ehrgeiz erwecken kann, kann es dazu f{\"u}hren, dass es nicht zu der erhofften Frustration kommt. Um diese Emotion trotzdem einleiten zu k{\"o}nnen, wird das Spiel an einigen Stellen so prepariert, dass der Proband durch Spielfehler den Draht ber{\"u}hrt und erneut von dem Startpunkt aus starten muss. Die empfundene Ungerechtigkeit und Ohnm{\"a}chtigkeit des Probanden soll als Indikator der Frustration dienen.
