\subsection{Fragebogen} \label{fragebogen-1}

\todo[inline]{Verantwortlich: Boris}

Ein Fragebogen kann als Formular definiert werden, das einen Satz von Fragen enthält, die für einen bestimmten Zweck definiert wurden \cite{gault-questionnaire}.
Hier dient der Fragebogen als Befragungsinstrument für die Realisierung unseres Projekts. Hierfür wurden für die Ausarbeitung die Hauptkomponenten definiert, die Auskunft über das Problem der Studie und über die Versuchsperson geben. 
Für diese Studie wurden zwei Arten von Fragebögen definiert. \\

Der erste Typ basiert auf einem allgemeinen Modell. Es enthält einen informativen Teil, der ein Text ist, der Informationen über das zu erreichende Ziel enthält. Der andere Teil besteht aus vier "Dropdown-Boxen" von dreizehn Optionen, die zwölf verschiedene Emotionen und einen als null oder neutral geltenden Zustand enthalten. Jede Dropdown-Box stellt ein Viertel der Zeit der Prüfung dar, der die Testperson unterzogen wurde. Sie muss zwischen diesen Optionen jeder Dropdown-Box wählen, was sie am meisten empfand, je nachdem, wann sie es fühlte, d.h. ob es das erste, zweite, dritte oder letzte Quartal der Zeit des Videos war, um die Emotion zu bewältigen.  Die Testperson ein anderes, wenn er seine Wahl ändern möchte, auch nach dem Anklicken des nächsten Buttons, er kann zum Fragebogen zurückkehren, indem er den Zurück-Button drückt und die gewünschten Änderungen vornimmt. Die Versuchsperson hat auch die Möglichkeit, eine andere Option zu wählen, wenn sie ihre Wahl ändern möchte. Auch nach dem Drücken der nächsten Taste kann er durch Drücken der Zurück-Taste zum Fragebogen zurückkehren und die gewünschten Änderungen vornehmen. \\

(Bild für Fragebogen) \\

Der zweite Typ ähnelt dem berühmten Modell von James Russels "circumplex" \cite{russel_1980}. Es ist ein klassisches Modell mit einer kreisförmigen Struktur, die auf zwei senkrechten Diagonalen ruht. Die vertikale Achse, die die Erregung darstellt, und die horizontale Achse, die die Valenz darstellt. Das Zentrum des Kreises stellt eine neutrale Valenz und ein mittleres Erregungsniveau dar. Andere Emotionen werden auf jeder Ebene des Kreises dargestellt.  Hier wird ein weniger bekanntes Modell verwendet, das "Self Assessment Manikin" (SAM). Es besteht aus drei Reihen mit je fünf Piktogrammen. Diese Piktogramme stellen den Zustand eines Gesichts nach verschiedenen Arten von Emotionen dar. So repräsentiert der erste Bereich die Wertigkeit, der zweite die Erregung und der dritte die Dominanz. Eine Erklärung zu jedem dieser Begriffe ist ebenfalls neben dem Fragebogen enthalten, um die Testpersonen über diese Wörter aufzuklären.  Unter jeder Fläche und in der Mitte jeder der beiden Flächen befindet sich ein Kontrollkästchen (Checkbox).  So muss die Testperson für jede Zeile das Checkbox auswählen, das ihrem emotionalen Zustand am besten entspricht. Er kann nur ein Checkbox pro Zeile markieren und hat auch die Möglichkeit wie bei dem ersten Model seine Wahl zu ändern. \\

(Bild für circumplex)