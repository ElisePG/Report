\subsubsection{Langeweile} \label{langeweile-1}

\todo[inline]{Verantwortlich: Boris\\}

Langeweile wird im Allgemeinen als eine Emotion wahrgenommen, die man erlebt, wenn man einen Zustand durchläuft, an dem man kein Interesse hat \cite{vodanovich_2003}. 
Dieses Gefühl charakterisiert Inaktivität, Unproduktivität und kann manchmal zu Melancholie oder Traurigkeit führen.
In dieser Studie wird das Erwachen der Langeweile durch eine grüne rotierende Scheibe mit einem schwarzen Strahl verursacht. \\

Für das zweite Prototyp wird zwei Szenarios benutzt um die Langweile bei Probanden zu erwecken.
Als ersten wurde eine 5 minütigen Video über die "Latente Steuer im Jahres Abschluss" von der Tax Universität benutzt um diese Emotion auszulösen. Dabei ging es um eine kurze Einführungsvideo über Auswirkung von Latentesteuer in der Jahresbilanz. 
Als zweites Szenario würde das "Peg-Turning“ Spiel benutzt. Sein Prinzip besteht darin, eine nach alle 15 Sekunden (oder etwas länger) drehende kreisförmige grüne Scheibe zuzuschauen. \\

(Bild für Langweile (Pec-turning game))\\
