\subsubsection{Langeweile} \label{langeweile-1}

\todo[inline]{Verantwortlich: Boris\\}

Ärger wird im Allgemeinen als eine Emotion wahrgenommen, die man erlebt, wenn man einen Zustand durchläuft, an dem man kein Interesse hat \cite{vodanovich_2003}. Dieses Gefühl charakterisiert Inaktivität, Unproduktivität und kann manchmal zu Melancholie oder Traurigkeit führen.
In dieser Studie wird das Erwachen der Langeweile durch eine grüne rotierende Scheibe mit einem schwarzen Strahl verursacht. \\

(Bild für langweile)\\

Bei unserer ersten Prototype wurde auch eine 5 minutigen Video über die "Latente Steuer im Jahres Abschluss" von der Tax Universität benutzt um diese Emotion bei Probanten auszulösen. Dabei ging es um eine kurze Einführungsvideo über Auswirkung von Latentesteuer in der Jahresbilanz. Die sollte durch seine sozusagen "nichts zu tun" mit dem vorgestellten Thema die Langweile erwecken.