\subsection{Vorverarbeitung} \label{vorverarbeitung-1}

Wie bereits in Kapitel \ref{vorverarbeitung-0} beschrieben, ist das Ziel der Vorverarbeitung die ”Verbesserung” der Daten für die nachfolgenden Schritte der ERC.
Im Rahmen des ELISE Projektes wurden Normalisierungstechniken auf dem gesamten Datensatz angewendet. 
Wir haben insbesondere die Standardnormalisierung verwendet, welche den Mittelwert der Daten auf Null setzt und die Einheitsvarianz ergibt \cite{grus15}. 
Die Formel für die Standardnormierung lautet:
\begin{equation} 
\Large{ {x'={\frac {x-{\overline {x}}}{\sigma }}} } 
\label{equ:norm} \end{equation} %\vspace{0.5cm}

wobei $ x $ ein Datenpunkt eines Sensorkanales, $ \overline{x} $ ist der Durchschnitt der Gesamtheit für diesen Sensorkanal und $ \sigma $ ist die entsprechende Standardabweichung. \\