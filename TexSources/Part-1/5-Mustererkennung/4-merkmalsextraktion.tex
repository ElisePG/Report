\subsection{Merkmalsextraktion} \label{merkmalsextraftion-1}

Wie bereits in Kapitel \ref{merkmalsextraktion-subsubsec} beschrieben, ist das Ziel der Merkmalsextraktion Charakteristiken und Merkmale in den Daten zu finden, die für das
Klassifizierungsproblem von möglichst hoher Relevanz sind. Im Rahmen des ELISE Projektes haben wir verschiedene Vorgehensweisen angewendet. Im den folgenden Unterkapiteln werden diese vorgestellt. \\



% Unterkapitel 
\subsubsection{Handgefertigte Merkmale} \label{hc-features-1-subsubsection}
Der handgefertigten Merkmal Ansatz (enlg. "hand-crafted features approach") besteht in der Berechnung relativ einfacher Merkmale von denen vermudetet wird, dass sie für das Klassifizierungsproblem der Eingangssignale relevant sein können. Diese Vorgehensweise hat den Vorteil des einfachen Aufbaus als auch der relativ geringen benötigten Rechenleistung, wobei potentiell gute Klassifizierungsergebnisse erwarten werden. \\






% Unterkapitel 
\subsubsection{Codebook Approach} \label{ca-1-subsubsection}

