\subsubsection{Datenerfassung} \label{datenerfassung-subsubsec}


\todo[inline]{Artur: Später mit dem Rest der Dokumentation abstimmen.}


Dieser Schritt der ERC beinhaltet die Auswahl sowie den Aufbau der Sensoren, die Messreihendurchführung um Daten zu erhalten und Etikett-Beschriftungstechniken.
All diese Details sind wichtig für die Emotionserkennung, damit relevante und möglichst fehlerfreie Daten von Versuchspersonen für die verschiedenen emotionalen Zustände gewonnen werden können.
Dies ist besonders wichtig, da der Datenerfassungsschritt der erste in der ERC ist und die Ergebnisse aller folgenden Schritte direkt von der Qualität des Datensatzes abhängen. \\



% Vorverarbeitung, um die Daten von Rauschen zu reinigen.