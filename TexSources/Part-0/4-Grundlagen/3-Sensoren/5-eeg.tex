\subsubsection{Elektroenzephalografie (EEG)} \label{grund-eeg-subsubsec}


Die Elektroenzephalografie (EEG) ist eine Methode zur grafischen Darstellung von Spannungsschwankungen an der Kopfoberfläche, und wird sowohl in der medizinischen Diagnostik, als auch der neurologischen Forschung angewendet. Das Gehirn besteht hauptsächlich aus Nervengewebe und bildet zusammen mit dem Rückenmark das Zentralnervensystem. Im Gehirn werden die hochdifferenzierten Sinneswahrnehmungen verarbeitet und komplexe Verhaltensweisen koordiniert. 1875 leitete der Forscher Richard Caton das erste Mal ein EEG ab und beobachtete in Tierversuchen die elektrischenPhänomene im Gehirn. Wenn Neuronen miteinander kommunizieren produzieren diese dabei elektrische Impulse. Einige Impulse können Forscher heutzutage messen, jedoch längst noch nicht alle. Durch externe (z.B. visuelle und auditive) und interne (z.B. sensorische, motorische und psychische) Reize werden unterschiedliche Potentiale an unterschiedlichen Orten im Gehirn generiert und können durch am Kopf angebrachte Elektroden gemessen werden. Visuelle Reize werden beispielsweise im Sehzentrum verarbeitet und lassen sich durch einen wiederkehrenden Reiz messen und können durch Aktivitäten in der spezifischen Gehirnregion eingeordnet werden. Hinter der Stirn lassen sich emotionale Signale messen, die sogennanten P300-Wellen (Ondes P300). Diese treten nach einem emotionalen Ereignis auf, und zwar exakt 300 Millisekunden später.  Da EEG ein günstiges, nicht-invasives und zuverlässiges Messsystem ist, eignet es sich sehr gut für die Messung von emotionalen Stimulierungen im Gehirn. 
Die Messung von Gehirnaktivitäten enthält keine lang andauernden Oszillationen. Durch Änderungen im Frequenzspektrum können Analysen Aussagen über den Bewusstseinszustand treffen. Aus diesem Grund werden Aktivitäten im EEG häufig in fünf unterschiedliche Frequenzbänder unterteilt. Die Anzahl von Bändern wie auch die genaue Einteilung werden von Autoren verschieden angegeben. Im Rahmen dieser Projektgruppe wurde folgende Aufteilung verwendet.

Delta-Wellen: < 4 Hz, treten in der Tiefschlafphase auf.

Theta-Wellen: 4-7 Hz, kommen bei Schläfrigkeit und in den leichten Schlafphasen vor 

Alpha-Wellen: 8-15 Hz, bei leichter Entspannung, bei geschlossenen Augen

Beta-Wellen: 16-31 Hz, aktive Konzentration, konstante Anspannung eines Muskels

Gamma-Wellen: >31 Hz, starke Konzentration, Lernprozesse
