\subsubsection{Galvanic Skin Response (GSR)} \label{grund-gsr-subsubsec}


Galvanic Skin Repsonse(GSR) bezeichnet ein kurzzeitige Änderung der elektrischen Hautleitfähigkeit, und wird auch gebräuchlicher weiße als elektrodermale Aktivität (EDA) bezeichnet. So können emotionale Reaktionen unbewusst zu einer erhöhten Schweißsekretion führen, wodurch die Hautleitfähigkeit für eine kurze Zeit erhöht wird. Da die Haut nicht der kognitiven Kontrolle unterliegt, sondern dem vegetativen Nervensystem, lassen sich mit Hilfe von GSR physiologische und psychologische Prozesse im Körper wahrnehmen und messen.  Die Haut ist funktionell das vielseitigste Organ des menschlichen Körpers und Schnittstelle zur Umgebung. Zusammen mit anderen Organen bildet sie für unseren Körper einen Schutz vor Umwelteinflüssen, besitzt eine Temperaturregulierung und verleiht dem Körper einen Tastsinn durch eine Vielzahl von unterschiedlichen Rezeptoren.  Der Körper ist mit über drei Millionen Schweißdrüsen übersät, die meisten befinden sich dabei auf Stirn, Fußsohlen, Händen und unter den Armen. Aus der Annahme, Lügen verstärkt eine sympatische Erregung, resultiert die Schlussfolgerung, dass sich das Verfahren GSR auch als Indikator für Falschaussagen beim Lügendetektor verwenden ließe. So entsteht die Frage, ob durch Videos, Bilder, Töne oder andere sensorische Anreize Emotionen stimuliert und gemessen werden können.
