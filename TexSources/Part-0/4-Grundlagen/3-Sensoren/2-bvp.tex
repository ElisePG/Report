\subsubsection{Blood Volume Pulse-Sensor (BVP)} \label{grund-bvp-subsubsec}


Mit einem BVP-Sensor wird der Blutvolumen-Puls gemessen, der mit der Herzrate in Korrelation
steht und somit als ein Zeichen für den Erregungsgrad fungiert. Der Sensor besitzt
eine rote oder grüne Lichtquelle (Emitter) und einen optischen, nicht-invasiven Sensor, der
die kardiovaskulären Änderungen durch unterschiedliche Lichtdurchlässigkeiten, auch Transluzenz
genannt, in der Arterie misst. 

Auch hier gibt es verschiedene Ansätze zum Messen des Blutvolumen-Pulses. Entweder wird die Lichtdurchlässigkeit einer Extremität (z.B. Finger oder Ohrläppchen) oder die Reflektion des Lichts in der Arterie gemessen. 

Wenn das Herz Blut durch die Arterien pumpt, wird die Lichtdurchlässigkeit undurchsichtiger und weniger Licht (durch eine größere Lichtabsorbtion) passiert die Arterie vom Licht-Emitter zum Sensor. Beim zweiten Verfahren wird mehr Licht durch die Arterie reflektiert
und trifft auf den Sensor.

Die ermittelten Werte sagen etwas über die Sensoren und biophysiologische Signale zur Emotionserkennung Gefäßsystem aus, beispielsweise ob Verkrampfungen oder Verspannungen vorliegen. Ebenfalls kann durch eine kleine Pulsamplitude auf eine Migräne geschlossen werden.