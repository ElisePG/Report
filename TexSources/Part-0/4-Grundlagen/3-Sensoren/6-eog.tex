\subsubsection{Elektrookulografie (EOG)} \label{grund-eog-subsubsec}


Die Elektrookulografie ist ein Messverfahren zur Ermittlung des Netzhautruhepotenzials und wird häufig in der Diagnostik zur Ermittlung von Erkrankungen, z.B. von Gleichgewichtsstörungen, Nystagmus (unkontrollierte rhythmische Augenbewegungen) oder Erkrankungen der Netzhaut, verwendet. Das Messverfahren wird auch Elektronystagmographie genannt. Bei der EOG-Messung  wird entweder die Bewegung der Augen, oder aber die Veränderungen des Ruhepotentials der Netzhaut gemessen. Mit dem Ruhepotenzial der Netzhaut ist eine permanent bestehende Spannungsdifferenz zwischen der Rückseite und der Vorderseite des Augapfels gemeint. Um das Ruhepotenzial zu messen, werden zwei Elektroden benötigt, die jeweils paarweise entweder rechts und links (in diesem Fall beider Augen) oder aber oberhalb und unterhalb eines Auges angebracht werden. Eine dritte Elektrode dient dabei lediglich als Referenz zur Messung der Differenzen. Dadurch lassen sich kleinste Augenbewegungen ermitteln. Das Ruhepotenzial verändert sich bei einer Augenbewegung, da die Vorderseite des Auges näher an eine der Elektroden heran kommt. Die Rückseite des Auges nähert sich dagegen der gegenüberliegenden Elektrode. Die Elektrookulografie wird in aller Regel ambulant durchgeführt und ist für den Patienten mit keinerlei Schmerzen, Risiken Sensoren und biophysiologische Signale zur Emotionserkennung oder Nebenwirkungen verbunden. Durch dieses Messverfahren können ebenfalls Blickbewegungsmessungen aufgezeichnet werden, wodurch der Blickverlauf einer Person sichtbar gemacht werden kann. Diese Methode wird als Eye-Tracking bezeichnet und ist eine experimentelle Methode zur Gewinnung von Emotionen und Gehirnaktivitäten. Durch Kombination weiterer biometrischer Sensoren, kann das menschliche Verhalten in Situationen in Virtual-Reality differenzierter analysiert und ausgewertet werden und könnte den Lern- und Bildungseffekt während Schulungen weiter fördern.