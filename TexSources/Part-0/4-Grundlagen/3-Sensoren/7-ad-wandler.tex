\subsubsection{Analog/Digital-Wandler} \label{grund-ad-wandler-subsubsec}


Ein Analog/Digital-Wandler diskretisiert ein zeit-kontinuierliches (also analoges) Eingangssignal in einzelne diskretere Abtastsignale. Es wird also ein digitaler Wert erstellt, der dem Prozessorkern verfügbar gemacht wird. Das Abtasttheorem von Nyquist/Shannon/Raabe besagt, dass ein analoges Signal mit mehr als dem doppelten seiner Frequenz abgetastet werden muss, um ein fehlerfreies digitales Signal zu erhalten. Die Auflösung eines AD-Wandlers gibt dessen Genauigkeit an. Mit einer höheren Auflösung können kleinere Spannungsunterschiede des Eingangssignales erkannt werden. Die Auflösung gibt also prinzipiell an, für welchen Spannungswert das LSB (Least Significant Bit) steht. Also ob dieses z.B. 10 mV oder 1 mV betraägt, wobei letzteres eine höhere Auflösung darstellen würde. Als Beispiel besitzt der Mikrocontroller ATmega328P eine Referenzspannung von Uref= 3,3 Volt und eine Auflösung von 10 Bit. Dadurch wird der analoge Wertebereich der elektrischen Größe in 210, also 1024 gleich große Abschnitte unterteilt. Man spricht von einer 10-Bit Wandlung beziehungsweise von einem 10-bit Wandler. 
