\subsubsection{Körpertemperatur-Sensor} \label{grund-temp-subsubsec}

\todo[inline,color=green!40]{RfP}

Die Temperatur ist die nach der Zeit am zweithäufigsten gemessene physikalische Größe.
Um Temperaturen messen zu können, benötigt man einen Sensor, der die Temperatur in eine
Spannung oder einen Strom umsetzt. Zudem benötigt man einen Analog-Digital-Wandler
(deutsch: AD-Wandler, englisch: analog-digital-converter = ADC), der das analoge Signal digitalisiert. Diese sind entweder direkt auf dem Sensor angebracht oder in dem Mikrocontroller
integriert. Natürlich gibt es eine Reihe von verschiedenen Temperatursensoren, die mit temperaturabhängigem Widerstand bis zum fertigen All-in-one-Bauteil mit digitalem Ausgang
ausgestattet sind. Mit dem Temperatursensor kann die Körperschalentemperatur gemessen werden, die sich je nach Gemütslage leicht verändern kann. Die Körperschale beinhaltet die stoffwechselarmen, peripheren Anteile der Extremitäten und die gesamte Haut. Von der Körperschalentemperatur wird die Körperkerntemperatur abgegrenzt. Das heißt, dass die Körperschalentemperatur sehr in Abhängigkeit von der Umgebungstemperatur schwankt und durch Konvektion und Konduktion von der Körperkerntemperatur beeinflusst wird.  Bei allen Sensoren sollte man auf die jeweilige Auflösung und Genauigkeit achten um zuverlässige Werte für die Emotionsforschung zu erhalten!