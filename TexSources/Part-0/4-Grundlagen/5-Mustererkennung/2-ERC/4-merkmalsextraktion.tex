\subsubsection{Merkmalsextraktion} \label{merkmalsextraktion-0}

Hier werden Charakteristiken und Merkmale in den Daten gesucht, die f{\"u}r das Klassifizierungsproblem von m{\"o}glichst hoher Relevanz sind. Alle nach dem Segmentierungsschritt extrahierte Daten-Zeitfenster kann durch einen Merkmalsvektor (engl. ``feature vector'') dargestellt werden. Mit Hilfe von Merkmalsvektoren kann ein Klassifikator dann einfacher trainiert werden als nur mit den Rohdaten.
Unser Fokus in der Mustererkennung lag vor allem auf der Merkmalsextraktion, da unsere Erfahrungen und fr{\"u}here Forschungsarbeiten gezeigt haben, dass die Wahl der Merkmale sehr wichtig f{\"u}r die endg{\"u}ltigen Klassifizierungsergebnisse sind. Dar{\"u}ber hinaus wurden noch keine state-of-the-art Merkmale f{\"u}r die Emotionserkennung mit dieser spezifischen Assoziation von eindimensionalen physiologischen Signalen gefunden. \\