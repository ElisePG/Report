\subsubsection{Segmentation} \label{segmentation-subsubsec}

\todo[inline,color=green!40]{Verantwortlich: Artur \\
- Bereit zum Korrekturlesen.}

Ziel dieses Schrittes ist es, Teile von Daten zu identifizieren, welche wichtige Informationen über die zu erkennenden Emotionen enthalten. 
Dies geschieht durch Filtern der Daten und Ausschließen von Segmenten, die für das Klassifizierungsproblem nicht relevant sind.
Zusätzlich wird die zu verarbeitende Datenmenge reduziert, indem Segmente eines Zeitfensters fester Größe aus den Daten extrahiert werden.
Diese Vorgeheisweise ist heute in der Praxis besonders wichtig, da sonst hardwarebedingte Einschränkungen die zu verarbeitende Datenmenge begrenzen könnten. \\