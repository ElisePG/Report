\subsubsection{Segmentation} \label{segmentation-0}

Ziel dieses Schrittes ist es, Teile von Daten zu identifizieren, welche wichtige Informationen {\"u}ber die zu erkennenden Emotionen enthalten. 
Dies geschieht durch Filtern der Daten und Ausschlie{\ss}en von Segmenten, die f{\"u}r das Klassifizierungsproblem nicht relevant sind.
Zus{\"a}tzlich wird die zu verarbeitende Datenmenge reduziert, indem Segmente eines Zeitfensters fester Gr{\"o}{\ss}e aus den Daten extrahiert werden.
Diese Vorgeheisweise ist heute in der Praxis besonders wichtig, da sonst hardwarebedingte Einschr{\"a}nkungen die zu verarbeitende Datenmenge begrenzen k{\"o}nnten. \\