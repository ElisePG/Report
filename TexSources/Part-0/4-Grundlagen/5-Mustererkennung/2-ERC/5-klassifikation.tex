\subsubsection{Klassifikation} \label{klassifikation-subsubsec}

\todo[inline]{Verantworltich: Artur \\}

Ziel des Klassifizierungsschritts ist es, ein Klassifizierungsmodell zu trainieren, das in der Lage ist, Objekte in den Daten (dargestellt durch ihren Merkmalsvektor) in die entsprechende Klasse zuzuordnen. \\

Der Datensatz der Merkmalsvektoren, der im vorherigen Schritt des ERC erhalten wurde, wird in einen Trainingsset (engl. "training set") und einen Testset (engl. "testing set") unterteilt, so dass alle Klassen in beiden Sets vorhanden sind. Mit dem Trainingsdaten wird ein Klassifikator erstellt und trainiert. Der so erhaltene Klassifikator wird dann anhand der Daten des Testsets ausgewertet. Es ist wichtig, dass die Trainings- und Testsets unterschiedlich sind (d.h. nicht die gleichen im Daten Trainings- und Testset verwenden), da es sonst in einer Überanpassung (engl. "overfitting") des Klassifikator resultieren kann. Eine Überpassung tritt auf, wenn ein Klassifikator zufällige Schwankungen oder Rauschen in den Trainingsdaten "zu gut" lernt und dann bei neuen, unbekannten Daten deutlich schlechter abschneidet. Der Grund hierfür ist, dass diese gelernten Schwankungen oder Rauschen in den Trainingsdaten keinerlei Relavanz für das eigentliche Klassifizierungsproblem haben. \\