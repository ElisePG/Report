\subsubsection{Datenerfassung} \label{datenerfassung-0}


\todo[inline,color=green!40]{Verantwortlich: Artur \\
@Kevin, bitte Korrekturlesen und mit Deinem Teil abstimmen.}


Dieser Schritt der ERC beinhaltet die Auswahl und den Aufbau der Sensoren, die Messreihendurchführung (um Daten zu erhalten) und Etikett-Beschriftungstechniken.
Ziel ist es relevante und möglichst fehlerfreie Daten von Versuchspersonen für die verschiedenen emotionalen Zustände zu gewinnen.
Der Datenerfassungsschritt ist besonders wichtig, da er der erste in der ERC ist und die Ergebnisse aller folgenden Schritte direkt von der Qualität des Datensatzes abhängen. \\