\subsection{Grundlagen der Mustererkennung} \label{grundlagen-mustererkennung}


Mustererkennung (enlg. ``pattern recognition") ist ein Unterthema des machinellen Lernens.
Das Ziel besteht darin, automatisierte Systeme zu entwerfen, die hoch abstrakte Muster in Daten erkennen k{\"o}nnen.
Konkret hei{\ss}t dies, dass man Maschinen beibringen m{\"o}chte komplexer Aufgaben zu l{\"o}sen, welche vom Menschen nahzu m{\"u}hlelos und nat{\"u}rlich erledigt werden k{\"o}nnen.
Typische Beispiele f{\"u}r die zahlreichen Anwendungsbereiche sind die Objekterkennung, Spracherkennung sowie die Erkennung und Verfolgung in Bildern. 
Die Emotionserkennung ist ein Anwendungsbereich der Mustererkennung.
Die Hauptidee hinter der L{\"o}sung eines Mustererkennung-Problems ist es, dieses als Klassifikationsproblem zu {\"u}bersetzen, wobei die zu erkennende Mustern die unterschiedliche Klassen bilden. 
Die vom Mustererkennungs-System eingegebenen Daten werden dann verarbeitet und der ``am n{\"a}chsten liegenden" Klasse zugeordnet.
Beispielsweise k{\"o}nnen bei der Emotionserkennung die Eingangsdaten Bilder oder physiologische Signale sein, die in verschiedene Klassen eingeteilt werden, welche jeweils einer Emotion entsprechen. \\

Ein wichtiger Teil eines jeden Mustererkennung-Problems ist der Lernansatz, mit welchem die Maschine lernen soll die Muster in den Daten zu erkennen. 
Traditionell werden zwei Ans{\"a}tze verwendet:

\begin{itemize}%[noitemsep]
  \item \underline{{\"u}berwachter Lernansatz:}
  Dieser Ansatz kann nur verwendet werden, wenn vor der Verarbeitung der Daten ein Datenbeschriftungsschritt durchgef{\"u}hrt wurde.
  In diesem Schritt wird jedem Element des Datensatzes ein Etikett (engl. ``label") zugewiesen, das angibt, welcher Klasse der jeweilige Datenpunkt zugeordnet werden kann.
  Die zus{\"a}tzlichen Informationen, die die Etiketten liefern, werden als Grundlage verwendet, um sie mit der Vorhersage des Systems zu vergleichen und zu korrigieren, wenn sie nicht gleich sind.

  \item \underline{Un{\"u}berwachter Lernansatz:}
  Dieser Ansatz wird verwendet, wenn keine Etiketten f{\"u}r die Daten vorhanden sind.
  Un{\"u}berwachte Lerntechniken zielen darauf ab, der Maschine beizubringen, Muster in den Daten selbst zu finden. 
  Sie werden meist verwendet, um Einblicke in Daten zu erhalten, deren Struktur unbekannt ist.
\end{itemize} %\vspace{0.5cm}


{\"u}berwachtes Lernen liefert aktuell weit bessere Ergebnisse, jedoch ist die Beschriftung mit Etiketten der Daten nicht immer einfach oder teilweise sogar gar nicht m{\"o}glich (z.B. wenn die Datenmenge sehr gro{\ss} ist oder wenn Unsicherheit {\"u}ber die Vergabe der Etikette besteht).
Aus diesem Grund w{\"a}chst das Interesse an un{\"u}berwachter Lernans{\"a}tzen.
Diese Ans{\"a}tze sind jedoch schwierig zu benutzen, da sie eine gro{\ss}e Menge an Daten voraussetzen.
Kompromisse sind mit semi-{\"u}berwachten Lernans{\"a}tzen m{\"o}glich, bei denen die Daten f{\"u}r einen Teil des Datensatzes (aber nicht f{\"u}r den ganzen Datensatz) mit Etiketten beschriftet und damit bekannt sind.
In diesem Fall kann eine Mischung aus {\"u}berwachten und Un{\"u}berwachter Techniken angewendet werden \cite{Zhu2008}. \\


Im Rahmen des ELISE-Projekts werden mit Hilfe von Mustererkennungsverfahren eindimensionale Zeitsignale von physiologischen Sensoren in Echtzeit f{\"u}r die Erkennung von drei Emotionen verarbeitet: Gl{\"u}ck, Frustation und Langeweile. Um den Emotionsklassifizierer aufzubauen, wird ein standardm{\"a}{\ss}iger, {\"u}berwachter Lernansatz namens Emotion Recognition Chain verwendet, der im folgendem Kapitel beschrieben wird. \\