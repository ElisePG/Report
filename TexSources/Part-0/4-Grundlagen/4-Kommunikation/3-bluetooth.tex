\subsection{Bluetooth} \label{grund-bluetooth-subsubsec}

\todo[inline]{Verantwortlich: Kevin, Jonas}

Der in den 1990er Jahren durch die Bluetooth Special Interest Group entwickelte Industriestandard (gemäß IEEE802.15.1) definiert die Datenübertragung zwischen Geräten mittels Funktechnik(WPAN). Mit Bluetooth können Geräte drahtlos und ohne direkte Sichtverbindung über kurze Distanzen miteinander kommunizieren. Der Name Bluetooth bezieht sich auf den dänischen König Harald II. Blaatand (übersetzt Blauzahn, englisch Bluetooth), der verfeindete Teile von Norwegen und Dänemark vereinte. So sollte auch diese Funktechnik die Computer- und Telekommunikations-Welt zusammenführen und dadurch eine Vielzahl unterschiedlicher Anschlüsse ablösen sollte. Bluetooth definiert einen vollständigen Protokollstapel bis zur Anwendungsschicht. Somit erfolgt die Zusammenarbeit der Geräte auf der Anwendungsebene und definiert zudem unterschiedliche  Profile für verschiedene Anwendungsbereiche, wie zum Beispiel Netzverwaltung, verbindungsorientierter oder –loser Dienst, Dienst-Abfragen, Telefondienste, usw..
Es gibt für Bluetooth eine Vielzahl unterschiedlicher Anwendungsfälle. Zum einen die Vernetzung mobiler Endgeräte, wie z.B. Smartphones, Ein- und Ausgabegeräte, PCs, Notebooks aber auch Mikrocontroller, zum Austausch von Daten und wichtiger Informationen. Zum anderen vor allem in der heutigen Zeit zwischen Smartphones und Audiogeräten. Die Übertragung läuft dabei über logische Kanäle (sogenannte Links) und erfolgt entweder asynchron, das heißt verbindungslos nach dem best effort-Prinzip, oder synchron, das heißt verbindungsbehaftet zur Echtzeitübertragung nach dem guaranteed timeliness-Prinzip. Die Kommunikation ist von Punkt zu Punkt, Ad-hoc- bis hin zu Piconetzen möglich und wird durch einen Master durch Vergabe von Sende-Zeitschlitzen (Zeitmultiplexverfahren), dem sogenannten Frequenzsprung-Spread-Spectrum-Verfahren (Frequency Hopping Spread Spectrum, FHSS), gesteuert. In dem Frequenzsprungverfahren beträgt die nominale Sprungrate 1600 Hops pro Sekunde. Aus der Sprungrate ergibt sich ein Zeitschlitz mit einer Länge von 625 %μs 
für jeden Slave, wobei eine Übertragung von allen Teilnehmern nur zu Beginn eines Zeitschlitzes gestartet werden darf. Die Frequenzfolge ist pseudozufällig und somit in jedem Piconetz unterschiedlich. Auf diese Weise soll der Betrieb von möglichst vielen unabhängigen Piconetzen mit hoher räumlicher Dichte unterstützt werden. Ein Bluetooth-Netzwerk besteht aus acht aktiven Teilnehmern (Master und maximal sieben aktive Slaves), welche über eine 3-Bit-Adresse angesprochen werden können. Die nicht aktiven Geräte können geparkt werden, die dennoch die Synchronisation mit dem Master halten und auf Anfrage im Netz aktiviert werden können. Über die zusätzliche 8-Bit-Adresse können bis zu 255 geparkte Slaves angesprochen werden. Bluetooth-Geräte können in mehreren Piconetzen angemeldet sein. Dadurch entstehen über das Gerät mehrere Piconetze, die ein Scatternet bilden, wie in Abbildung 3.11 veranschaulicht.
