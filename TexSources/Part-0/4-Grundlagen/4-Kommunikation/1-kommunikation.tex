\subsection{Kommunikation} \label{grund-kommunikation}

\todo[inline]{Verantwortlich: Kevin, Jonas}

Zur {\"u}bertragung der gewonnenen Sensordaten, wie auch zur Kommunikation der einzelnen Teilkomponeneten untereinander, wurden verschiedene Technologien und Protokolle evaluiert. \\

Die Implementierung erfolgte schlussendlich mit UDP Broadcasts und einem rudiment{\"a}ren Kommunikationsprotokoll.
F{\"u}r die {\"u}bertragung war es wichtig, dass die Implementierung sowohl auf Seiten der Unreal Engine, als auch beim Mikrocontroller und bei der Kontrollanwendung einfach und mit geringem Arbeitsaufwand m{\"o}glich ist.
Zugleich sollte auf eine einfache Wartbarkeit und die M{\"o}glichkeit die Verbindung ``live'' zu debuggen geachtet werden. \\

Die ersten Tests erfolgten mit einer einfachen Ausgabe der Sensordaten {\"u}ber eine USB-Serial Verbindung.
Da jedoch der Sensor abgesetzt und drahtlos Daten senden sollte, wurde dann beim ESP32 UDP Broadcasts als Kommunikationsweg gew{\"a}hlt. Neben der einfachen Implementierung auf allen Teilkomponenten ist hierbei auch die M{\"o}glichkeit einer 1:n Kommunikation gegeben.
In Zukunft sollte dar{\"u}ber nachgedacht werden, Broadcasts durch Multicasts zu ersetzen, da dadurch in einem n:m Szenario eine einfachere Separierung der Datenstr{\"o}me der einzelnen Sensoren erfolgen kann. \\

Jeder Teilnehmer wird durch eine einzigartige ID identifiziert, die in jedem Paket mitgesendet wird. \\

(BILD EINF{\"u}GEN UNTERSCHIEDLICHE DATENPAKETE) \\

Der Sensor sendet hierbei nicht f{\"u}r jeden Datenwert ein Paket, sondern aggregiert immer 75 Datenwerte in ein Paket. Dabei werden die einzelnen Kan{\"a}le in 4 Pakete aufgeteilt.
Paket 1 enth{\"a}lt die Sensordaten f{\"u}r Temperatur, GSR und EEG1. 
Paket 2 die f{\"u}r. 
Paket 3 f{\"u}r. 
Paket 4 zuletzt dann die Daten f{\"u}r IR-RAW. Diese Aggregierung und Aufteilung wurde durch Instabilit{\"a}ten in Sensor Netzwerkstack erforderlich.
Trotzdem ist durch diese L{\"o}sung weiterhin eine Datenrate von 250 Samples/Sekunde m{\"o}glich. Diese liegt weit {\"u}ber dem theoretischen Minimum, was f{\"u}r eine verlustfreie Rekonstruktion des Signals n{\"o}tig w{\"a}re, so dass auch weiterhin die Kommunikation robust gegen{\"u}ber St{\"o}reinfl{\"u}ssen ist.