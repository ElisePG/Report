\subsection{Sensoren und biophysiologische Signale zur Emotionserkennung} \label{grund-sensoren}

\todo[inline]{Verantwortlich: Kevin}

Emotionen aus biophysiologischen Signalen abzuleiten ist eine neuere und weitere Entwicklung.
Durch das zentrale Nervensystem gesteuerte biologische Reaktionen des Körpers sind Emotionen nur teilweise oder gar nicht der Kontrolle unseres Bewusstseins unterlegen.  Aus diesem Grund besteht eine direkte und unverfälschte Verbindung zu unseren Emotionen, die uns die Möglichkeit gibt, einen unmittelbaren Einblick in den menschlich-affektiven Zustand zu gewähren. Bildgebende Verfahren sind dabei immer noch, insbesondere die nichtinvasiven Methoden der funktionellen Magnetresonanztomographie, unabdingbar, um einen Blick in das emotionsverarbeitende menschliche Gehirn zu ermöglichen.  Jedoch überwiegen in der Emotionsforschung die Nachteile der Magnetresonanztomographie durch zum Beispiel Klaustrophobie, Kosten, Dauer, Verwackelgefahr und Lautstärke, wodurch neue Systeme zur Kompensierung der Nachteile durch neue Sensorik entwickelt werden müssen. Unser Körper sendet die gemessenen Signale ohne Unterbrechung, so dass man einen kontinuierlichen Signalstrom erhält. Jedoch reagiert der Mensch sehr individuell auf Emotionen, was es sehr schwierig macht, allgemeingültige Regeln zu finden. Unbekannte Erlebnisse werden viel eindringlicher erlebt, eher bekannte Situationen im Gegenzug schwächer. Auch die Umgebung, das Alter und die Erfahrungen haben einen wesentlichen Einfluss auf die Empfindung. Als Biosignale werden alle physikalisch messbaren und kontinuierlich oder nahezu kontinuierlich registrierbaren Körperfunktionen bezeichnet. Hierbei unterscheidet man direkte bioelektrische Signale (z.B. Herzschlag, Hirnaktivität), indirekte bioelektrische Signale (z.B. Hautleitfähigkeit) und nicht elektrische Signale (z.B. Blutdruck, Atemfrequenz). Für die Aufzeichnung des Korpus der Emotionsforschung liegen Signale der Sensoren Körpertemperatur, BVP (Blood Volume Pulse), SpO2 (Sauerstoffsättigung), GSR (Galvanic Skin Response), EEG (Elektroenzephalografie) und EOG (Elektrookulografie) vor.

% Unterkapitel
\subsubsection{Körpertemperatur-Sensor} \label{grund-temp-subsubsec}

\todo[inline,color=green!40]{RfP}

Die Temperatur ist die nach der Zeit am zweithäufigsten gemessene physikalische Größe.
Um Temperaturen messen zu können, benötigt man einen Sensor, der die Temperatur in eine
Spannung oder einen Strom umsetzt. Zudem benötigt man einen Analog-Digital-Wandler
(deutsch: AD-Wandler, englisch: analog-digital-converter = ADC), der das analoge Signal digitalisiert. Diese sind entweder direkt auf dem Sensor angebracht oder in dem Mikrocontroller
integriert. Natürlich gibt es eine Reihe von verschiedenen Temperatursensoren, die mit temperaturabhängigem Widerstand bis zum fertigen All-in-one-Bauteil mit digitalem Ausgang
ausgestattet sind. Mit dem Temperatursensor kann die Körperschalentemperatur gemessen werden, die sich je nach Gemütslage leicht verändern kann. Die Körperschale beinhaltet die stoffwechselarmen, peripheren Anteile der Extremitäten und die gesamte Haut. Von der Körperschalentemperatur wird die Körperkerntemperatur abgegrenzt. Das heißt, dass die Körperschalentemperatur sehr in Abhängigkeit von der Umgebungstemperatur schwankt und durch Konvektion und Konduktion von der Körperkerntemperatur beeinflusst wird.  Bei allen Sensoren sollte man auf die jeweilige Auflösung und Genauigkeit achten um zuverlässige Werte für die Emotionsforschung zu erhalten!

% Unterkapitel
\subsubsection{Blood Volume Pulse-Sensor (BVP)} \label{grund-bvp-subsubsec}


Mit einem BVP-Sensor wird der Blutvolumen-Puls gemessen, der mit der Herzrate in Korrelation
steht und somit als ein Zeichen für den Erregungsgrad fungiert. Der Sensor besitzt
eine rote oder grüne Lichtquelle (Emitter) und einen optischen, nicht-invasiven Sensor, der
die kardiovaskulären Änderungen durch unterschiedliche Lichtdurchlässigkeiten, auch Transluzenz
genannt, in der Arterie misst. 

Auch hier gibt es verschiedene Ansätze zum Messen des Blutvolumen-Pulses. Entweder wird die Lichtdurchlässigkeit einer Extremität (z.B. Finger oder Ohrläppchen) oder die Reflektion des Lichts in der Arterie gemessen. 

Wenn das Herz Blut durch die Arterien pumpt, wird die Lichtdurchlässigkeit undurchsichtiger und weniger Licht (durch eine größere Lichtabsorbtion) passiert die Arterie vom Licht-Emitter zum Sensor. Beim zweiten Verfahren wird mehr Licht durch die Arterie reflektiert
und trifft auf den Sensor.

Die ermittelten Werte sagen etwas über die Sensoren und biophysiologische Signale zur Emotionserkennung Gefäßsystem aus, beispielsweise ob Verkrampfungen oder Verspannungen vorliegen. Ebenfalls kann durch eine kleine Pulsamplitude auf eine Migräne geschlossen werden.

% Unterkapitel
\subsubsection{Messen der Sauerstoffs{\"a}ttigung (SpO2)} \label{grund-spo2-subsubsec}




% Unterkapitel
\subsubsection{Galvanic Skin Response (GSR)} \label{grund-gsr-subsubsec}


Galvanic Skin Repsonse(GSR) bezeichnet ein kurzzeitige Änderung der elektrischen Hautleitfähigkeit, und wird auch gebräuchlicher weiße als elektrodermale Aktivität (EDA) bezeichnet. So können emotionale Reaktionen unbewusst zu einer erhöhten Schweißsekretion führen, wodurch die Hautleitfähigkeit für eine kurze Zeit erhöht wird. Da die Haut nicht der kognitiven Kontrolle unterliegt, sondern dem vegetativen Nervensystem, lassen sich mit Hilfe von GSR physiologische und psychologische Prozesse im Körper wahrnehmen und messen.  Die Haut ist funktionell das vielseitigste Organ des menschlichen Körpers und Schnittstelle zur Umgebung. Zusammen mit anderen Organen bildet sie für unseren Körper einen Schutz vor Umwelteinflüssen, besitzt eine Temperaturregulierung und verleiht dem Körper einen Tastsinn durch eine Vielzahl von unterschiedlichen Rezeptoren.  Der Körper ist mit über drei Millionen Schweißdrüsen übersät, die meisten befinden sich dabei auf Stirn, Fußsohlen, Händen und unter den Armen. Aus der Annahme, Lügen verstärkt eine sympatische Erregung, resultiert die Schlussfolgerung, dass sich das Verfahren GSR auch als Indikator für Falschaussagen beim Lügendetektor verwenden ließe. So entsteht die Frage, ob durch Videos, Bilder, Töne oder andere sensorische Anreize Emotionen stimuliert und gemessen werden können.


% Unterkapitel
\subsubsection{Elektroenzephalografie (EEG)} \label{grund-eeg-subsubsec}




% Unterkapitel
\subsubsection{Elektrookulografie (EOG)} \label{grund-eog-subsubsec}


Die Elektrookulografie ist ein Messverfahren zur Ermittlung des Netzhautruhepotenzials und wird häufig in der Diagnostik zur Ermittlung von Erkrankungen, z.B. von Gleichgewichtsstörungen, Nystagmus (unkontrollierte rhythmische Augenbewegungen) oder Erkrankungen der Netzhaut, verwendet. Das Messverfahren wird auch Elektronystagmographie genannt. Bei der EOG-Messung  wird entweder die Bewegung der Augen, oder aber die Veränderungen des Ruhepotentials der Netzhaut gemessen. Mit dem Ruhepotenzial der Netzhaut ist eine permanent bestehende Spannungsdifferenz zwischen der Rückseite und der Vorderseite des Augapfels gemeint. Um das Ruhepotenzial zu messen, werden zwei Elektroden benötigt, die jeweils paarweise entweder rechts und links (in diesem Fall beider Augen) oder aber oberhalb und unterhalb eines Auges angebracht werden. Eine dritte Elektrode dient dabei lediglich als Referenz zur Messung der Differenzen. Dadurch lassen sich kleinste Augenbewegungen ermitteln. Das Ruhepotenzial verändert sich bei einer Augenbewegung, da die Vorderseite des Auges näher an eine der Elektroden heran kommt. Die Rückseite des Auges nähert sich dagegen der gegenüberliegenden Elektrode. Die Elektrookulografie wird in aller Regel ambulant durchgeführt und ist für den Patienten mit keinerlei Schmerzen, Risiken Sensoren und biophysiologische Signale zur Emotionserkennung oder Nebenwirkungen verbunden. Durch dieses Messverfahren können ebenfalls Blickbewegungsmessungen aufgezeichnet werden, wodurch der Blickverlauf einer Person sichtbar gemacht werden kann. Diese Methode wird als Eye-Tracking bezeichnet und ist eine experimentelle Methode zur Gewinnung von Emotionen und Gehirnaktivitäten. Durch Kombination weiterer biometrischer Sensoren, kann das menschliche Verhalten in Situationen in Virtual-Reality differenzierter analysiert und ausgewertet werden und könnte den Lern- und Bildungseffekt während Schulungen weiter fördern.

% Unterkapitel
\subsubsection{Analog/Digital-Wandler} \label{grund-ad-wandler-subsubsec}


