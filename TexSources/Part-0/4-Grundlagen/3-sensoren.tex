\subsection{Sensoren und biophysiologische Signale zur Emotionserkennung} \label{grund-sensoren}

\todo[inline]{Verantwortlich: Kevin}

Emotionen aus biophysiologischen Signalen abzuleiten ist eine neuere und weitere Entwicklung.
Durch das zentrale Nervensystem gesteuerte biologische Reaktionen des Körpers sind Emotionen nur teilweise oder gar nicht der Kontrolle unseres Bewusstseins unterlegen.  Aus diesem Grund besteht eine direkte und unverfälschte Verbindung zu unseren Emotionen, die uns die Möglichkeit gibt, einen unmittelbaren Einblick in den menschlich-affektiven Zustand zu gewähren. Bildgebende Verfahren sind dabei immer noch, insbesondere die nichtinvasiven Methoden der funktionellen Magnetresonanztomographie, unabdingbar, um einen Blick in das emotionsverarbeitende menschliche Gehirn zu ermöglichen.  Jedoch überwiegen in der Emotionsforschung die Nachteile der Magnetresonanztomographie durch zum Beispiel Klaustrophobie, Kosten, Dauer, Verwackelgefahr und Lautstärke, wodurch neue Systeme zur Kompensierung der Nachteile durch neue Sensorik entwickelt werden müssen. Unser Körper sendet die gemessenen Signale ohne Unterbrechung, so dass man einen kontinuierlichen Signalstrom erhält. Jedoch reagiert der Mensch sehr individuell auf Emotionen, was es sehr schwierig macht, allgemeingültige Regeln zu finden. Unbekannte Erlebnisse werden viel eindringlicher erlebt, eher bekannte Situationen im Gegenzug schwächer. Auch die Umgebung, das Alter und die Erfahrungen haben einen wesentlichen Einfluss auf die Empfindung. Als Biosignale werden alle physikalisch messbaren und kontinuierlich oder nahezu kontinuierlich registrierbaren Körperfunktionen bezeichnet. Hierbei unterscheidet man direkte bioelektrische Signale (z.B. Herzschlag, Hirnaktivität), indirekte bioelektrische Signale (z.B. Hautleitfähigkeit) und nicht elektrische Signale (z.B. Blutdruck, Atemfrequenz). Für die Aufzeichnung des Korpus der Emotionsforschung liegen Signale der Sensoren Körpertemperatur, BVP (Blood Volume Pulse), SpO2 (Sauerstoffsättigung), GSR (Galvanic Skin Response), EEG (Elektroenzephalografie) und EOG (Elektrookulografie) vor.

% Unterkapitel
\subsubsection{K{\"o}rpertemperatur-Sensor} \label{grund-temp-subsubsec}




% Unterkapitel
\subsubsection{Blood Volume Pulse-Sensor (BVP)} \label{grund-bvp-subsubsec}




% Unterkapitel
\subsubsection{Messen der Sauerstoffs{\"a}ttigung (SpO2)} \label{grund-spo2-subsubsec}




% Unterkapitel
\subsubsection{Galvanic Skin Response (GSR)} \label{grund-gsr-subsubsec}




% Unterkapitel
\subsubsection{Elektroenzephalografie (EEG)} \label{grund-eeg-subsubsec}




% Unterkapitel
\subsubsection{Elektrookulografie (EOG)} \label{grund-eog-subsubsec}




% Unterkapitel
\subsubsection{Analog/Digital-Wandler} \label{grund-ad-wandler-subsubsec}


Ein Analog/Digital-Wandler diskretisiert ein zeit-kontinuierliches (also analoges) Eingangssignal in einzelne diskretere Abtastsignale. Es wird also ein digitaler Wert erstellt, der dem Prozessorkern verfügbar gemacht wird. Das Abtasttheorem von Nyquist/Shannon/Raabe besagt, dass ein analoges Signal mit mehr als dem doppelten seiner Frequenz abgetastet werden muss, um ein fehlerfreies digitales Signal zu erhalten. Die Auflösung eines AD-Wandlers gibt dessen Genauigkeit an. Mit einer höheren Auflösung können kleinere Spannungsunterschiede des Eingangssignales erkannt werden. Die Auflösung gibt also prinzipiell an, für welchen Spannungswert das LSB (Least Significant Bit) steht. Also ob dieses z.B. 10 mV oder 1 mV betraägt, wobei letzteres eine höhere Auflösung darstellen würde. Als Beispiel besitzt der Mikrocontroller ATmega328P eine Referenzspannung von Uref= 3,3 Volt und eine Auflösung von 10 Bit. Dadurch wird der analoge Wertebereich der elektrischen Größe in 210, also 1024 gleich große Abschnitte unterteilt. Man spricht von einer 10-Bit Wandlung beziehungsweise von einem 10-bit Wandler. 
