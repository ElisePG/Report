\todo[inline,color=green!40]{Verantwortlich: Arnaud\\
- RfP}

Mit diesem Kapitel werden Einblicke in die notwendige Grundlagen zur Realisierung unserer Arbeit gegeben. 
Zuerst werden wir uns mit der Thematik Emotion (Definition und Klassifikation) beschäftigen. 
Im Weiteren werden ``Virtual Reality'' ( beziehungsweise HTC Vive VR-Brille) genauer erklärt. 
Ebenfalls kommt eine Erläuterung der Sensoren und Biophysiologische Signale, welche f{\"u}r unsere Arbeit von Bedeutung sind. 
Näher wird auf die Kommunikation zwischen verschiedenen Sensoren eingegangen.
Dazu kommt auch eine Erklärung wie die Datenerfassung(Messreihe) stattfinden sollte.
Danach befassen wir uns mit der Grundlagen der Mustererkennung. 
Ein Berichtsteil {\"u}ber die ``Emotion Recognition chain'' schlie{\ss}t das Kapitel ``Grundlagen'' ab.


% Unterkapitel
\subsection{Definition von Emotionen} \label{definition-emotionen}


\todo[inline]{Verantwortlich: Arnaud}


Der Begriff ``Emotion'' wird zwar weltweit  (in fast allen Sprachen) und von alle Menschen verwendet (die soziale oder intellektuelle Ebene spielt keine Rolle), ist aber relativ schwer zu definieren. 
Dieses Paradox wurde bereits im folgenden Zitat explizit erwähnt: 
``Everybody knows what an emotion is, until asked to give a definition.''\cite{fehr_russel_1984} von Fehr und Russell, zwei amerikanischer Emotionsforscher (Psychologe). 
Noch überraschender ist die Schwankung in der Definition dieses Begriffs ``Emotion'' im Laufe der Zeit: Allein die englischsprachige Experimentalpsychologie\cite{plamper12} liefert zwischen 1872 und 1980 mehr als 92 verschiedene Definitionen. 
Wir verstehen daher, dass es schwierig wäre, zu versuchen, diese Definition zu formulieren.  
Wie können wir also die Schwierigkeit, eine Definition für einen solchen gemeinsamen Begriff zu finden, erklären? 
Man sollte nicht vergessen werden, dass es sich um einen eher abstrakten Begriff handelt und daher ist die Emotion sehr subjektiv. 
Neben diesem abstrakten Aspekt ist auch anzumerken, dass sich der Begriff der Emotion auch auf viele Bereiche bezieht, die sich ebenso voneinander unterscheiden wie sie variieren: z.B. Literatur, Philosophie, Psychologie usw. 
Neben diesem multidisziplinären Charakter, der die Pluralität der Vorstellungen und Ansätze in jedem Definitionsversuch erklären könnte, ist es auch notwendig, die Variationen von Sprachen, Perioden und sogar Kulturen hinzuzufügen. \\


Es wird uns allein mit dieser Arbeit (und das ist auch nicht von uns erwartet) nicht möglich sein, alle Fragen im Zusammenhang mit der Definition des Begriffs ``Emotion'' zu betrachten, aber wir werden einige konkrete Beispiele vorstellen, um die Komplexität zu veranschaulichen, die sich aus der Suche nach einer Definition von Emotion ergeben kann.  
Die antiken Philosophen\cite{geslin13} waren die ersten, die sich mit Emotionen und ihren Einflüssen auf den Alltag beschäftigen haben. 
Tatsächlich nahmen Stoiker wie Zeno und Plato bereits 370 n. Chr. Emotionen als eine Krankheit der Seele wahr, die für sie ein Hindernis für denjenigen war, der denken wollte. 
Platon geht mit seiner Allegorie von der Höhle tiefer, indem er alles was emotional ist  mit alle vernünftig (verständlich) kontrastiert: das heiß Emotion und Vernunft gehen nicht zusammen. 
Descartes und Aristoteles vervollständigen Platons Beobachtungen, indem sie in die Analyse von Emotionen eine Dualität (positiv und negativ) der Perzeption einbringen. 
Aristoteles glaubt, dass alles, was das Leben auf positive Weise beeinflusst, durch positive Emotionen bedingt ist, und Descartes glaubt, dass Emotionen für unser Überleben unerlässlich sind und dass die Schwäche eines Menschen eng mit der Fähigkeit der Seele verbunden ist, seine Emotionen zu kontrollieren. 
Charles Darwin in seinem präsentierte weitere ebenso faszinierende neue Elemente über Emotionen, ohne den bisherigen Beobachtungen zu widersprechen\cite{darwin1872}. 
Er verallgemeinert die Emotionen für alle Kulturen und fand sogar ähnlichkeit mit Tiere. 
Der Darwin präsentiert  Emotionen  als Körpersignale (oder Reaktionen) auf äußere Handlungen(Externe Ereignisse), begleitet von spezifischen körperlichen Äußerungen wie: Gesichtsausdrücke, Gesten und oft Geräusche, die alle  spezifisch je nach Emotionen sind. \\


So können wir weiterhin andere berühmte wissenschaftliche Namen wie William James\cite{james1884}, Walter Cannon\cite{cannon32}, Stanley Scharter\cite{schachter59} usw. nennen, die zu unterschiedlichen Zeiten und an unterschiedlichen Orten das Thema untersucht haben, mit ebenso relevanten wie unterschiedlichen Schlussfolgerungen, aber die Beobachtung bleibt die gleiche, die Definition bleibt unbeständig.
Neben den Schwierigkeiten, die sich aus den oben genannten unterschiedlichen Wahrnehmungen ergeben, trägt die  tägliche missbräuchliche Verwendung bestimmter Begriffe wie Gefühl, Affekt, Stimmung, Gefühl (um nur einige zu nennen) als Synonym für Emotionen dazu bei  mehr Verwirrung in das Verständnis von Emotionen. Diese Begriffe beziehen sich zwar auf Gemütszustände, sind aber jedoch nicht Synonym von Emotion\cite{plamper12}.
Trotz des fehlenden Konsenses über die Frage der Definition dieses Begriffs gibt es jedoch Elemente, die in allen Definitionsversuchen wiederkehren, nämlich: 

\begin{itemize} \setlength\itemsep{-0.15cm}
  \item das Vorhandensein eines Auslösers;
  \item den psychischen Zustand des Subjekts;
  \item einen bestimmten Körperausdruck;
  \item eine physiologische Reaktion (Herzfrequenz, Atmung,Schwitzen ...);
  \item eine bestimmte Qualität, Intensität und Dauer.
\end{itemize}


Basierend auf den obigen Elementen können wir daher zu dem Schluss kommen, dass wir zwar nicht genau definieren können, was eine Emotion ist, können wir aber jedoch jede davon mit Bestimmte spezifisch  Element in Verbindung setzen: einen Stimulus (der der Auslöser der Emotion ist), eine physiologische und körperliche Reaktion. 
Die Existenz mehrerer verschiedener Emotionen lässt sich gerade wegen der Spezifität dieser Elemente vermuten. 
Im nächsten Teil werden wir versuchen, die verschiedenen Emotionen zu klassifizieren, mit einem besonderen Fokus auf diejenigen, die wir zu induzieren versuchen. \\


Wie bei der Definition von Emotionen gibt es keine einheitliche Klassifizierung von Emotionen. 
Und wie bei den versuchten Definitionen gab es im Laufe der Zeit mehrere vorgeschlagene Klassifikationen, die sich auch im Laufe der Zeit entwickelt haben. 
Diese Mehrheit von klassifikationen ist vermutlich eine Folge des Mangels an Konsens über die Definition selbst. 
Es gibt jedoch zwei Arten von Klassifikationen, die vor allem unterscheidet und durchgesetzt haben: eine dimensionale Klassifikation\cite{geslin13} und eine kategorische Klassifikation\cite{basic_emotions_theories}.




\subsubsection{Kategoriale Klassifikation} \label{kategoriale-klassifikation}

Angetrieben von Charles Darwin, (der auch der erste war, der grundlegende Emotionen identifizierte) diese Vorgehensweise ist die von Emotionsforscher  mit einer evolutionären Ansatz. 
Diese Klassifikation basiert auf der Existenz von eine Reihe von sogenannten Grundemotionen oder primäre Emotionen. 
Diese  Emotionen wäre in einer begrenzten Anzahl und ihre Entwicklung wurde stark von der Evolution  beeinflusst. 
Wie viele Dinge in Bezug auf die Emotionen in allgemein, variiert die Anzahl von Grundemotionen von einem Emotionsforscher zu einem anderen (siehe Tabelle \ref{vergleich-basisemotionen}).


\begin{table}[h] \centering
\begin{tabular}{| p{5cm} | p{11cm} |}
\hline
\textbf{Forscher} & \textbf{Basisemotionen} \\ \hline
Gray (1982) & Furcht, Freude, Ärger \\ \hline
Panksepp (1982) & Furcht, Erwartung, Ärger, Panik \\  \hline
Tomkins (1984) & Furcht, Freude, Ärger, Verzweiflung, Ekel, Überraschung, Interesse, Scham, Zufriedenheit \\  \hline
Plutchik (1980) & Furcht, Freude, Ärger, Traurigkeit, Ekel, Überraschung, Akzeptanz, Erwartung \\  \hline
Arnold (1960) & Furcht, Liebe, Ärger, Traurigkeit, Hass, Hoffnung, Begehren, Mut, Niedergeschlagenheit, Verzweiflung, Widerwille \\  \hline
Oatley \& Johnson-Laird (1987) & Furcht, Glück, Ärger, Traurigkeit, Ekel \\  \hline
Ekman (1982) & Furcht, Freude, Ärger, Traurigkeit, Ekel, Überraschung \\  \hline
Izard (1987) & Furcht, Freude, Ärger, Traurigkeit, Ekel, Überraschung, Interesse,Verachtung \\ \hline
\end{tabular} \caption[Vergleich einige Basisemotions-Theorien]{ Vergleich einige Basisemotions-Theorien\cite{basic_emotions_theories}. } \label{vergleich-basisemotionen}
\end{table}



Obwohl sie sich nicht über die Anzahl der Grundemotionen einig sind, sind sich die Forscher einig über die Existenz komplexerer Emotionen, die die Kombination mehrerer primärer Emotionen wären. 
Basierend auf diesem Ansatz wurden mehrere Theorien entwickelt, um Emotionen darzustellen, von denen eine der bekannte das Plutchik-Modell ist. 
Plutchiks Modell verwendet ein radförmiges Diagramm (siehe Abbildung \ref{plutchik}) und verschiedene Farben, um jede Emotion darzustellen. \\


\begin{figure}[h]
\includegraphics[width=\textwidth]{Images/plutchik.png} 
\vspace{-0.3cm} 
\caption{ Einordnung von Emotion nach Plutchik. }
\label{plutchik} 
\end{figure}


Diese Theorie fundiert sich auf  das acht primäre Emotionen mit einer klare trennung (bzw unterschied) zwischen  primären, sekundären und tertiären Emotionen.
Die verbindet jede primäre Emotion mit spezifischen Motivationssystem und Verhaltendstedenze zur Bewältigung grundlegender adaptiver Probleme (siehe Tabelle \ref{verhalten-funktion}). 


\begin{table}[h] \centering
\begin{tabular}{| p{5.1cm} | p{5.1cm} | p{5.1cm} |}
\hline
\textbf{Subjektiv} & \textbf{Verhalten} & \textbf{Funktion} \\ \hline
Angst, Entsetzen & Rückzug, Flucht  & Schutz \\ \hline
Ärger, Wut & Angriff, Beißen & Zerstörung \\ \hline
Freude, Ekstase & Paarung, Besitz & ergreifen Fortpflanzung \\ \hline
Traurigkeit, Trauer & Weinen, Bitte um Hilfe & Reintegration \\ \hline
Akzeptanz, Anbetung, Vertrauen & Paarbildung, Pflege & Zusammengehörigkeit, Bindung \\ \hline
Ekel, Abscheu & Sich übergeben & Ablehnung, Zurückweisung \\ \hline
Erwartung, Antizipation & Untersuchen & Exploration, Erkundung \\ \hline
Überraschung & Innehalten, Einfrieren & Orientierung \\ \hline
\end{tabular} \caption[ Einige Basisemotionen jeweils mit Verhalten und Funktio ]{ Einige Basisemotionen jeweils mit Verhalten und Funktion\cite{basic_emotions_theories}. } \label{verhalten-funktion}
\end{table}






\subsubsection{Kategoriale Ansatz} \label{kategoriale-ansatz}

Dieser von Wundt\cite{basic_emotions_theories} initiierte Ansatz geht von dem Gedanken aus, dass die emotionale Erfahrung in einem mehrdimensionalen Raum dargestellt werden kann. 
Dieser Zerlegung der emotionale Erfahrung  sollte es ermöglichen, eine genaue Analogie zwischen Emotion und Körperausdruck (Gesichtsausdrücke) herzustellen. 
Dieser Ansatz wurde  daher den Vorteil, dass sie die Tür zu einer möglichen Quantifizierung der emotionalen Erfahrung öffnet. 
Die Idee von Wundt (siehe Abbildung \ref{wundt})  war einer dreidimensionalen Zerlegung(Lust-Unlust, Spannung-Entspannung (Lösung), Beruhigung (Ruhe)-Erregung siehe) der emotionalen Erfahrung. 


\begin{figure}[h]
\includegraphics[width=\textwidth]{Images/wundt.png} 
\vspace{-0.3cm} 
\caption[Einordnung von emotionale Erfahrungsprozess nach Wundt.]{ Einordnung von emotionale Erfahrungsprozess nach Wundt\cite{basic_emotions_theories}. }
\label{wundt} 
\end{figure}


Die Frage nach der Anzahl der Dimensionen, die zur Darstellung der emotionalen Erfahrung notwendig sind, wird jedoch zu mehreren Theorien führen. 
Allerdings schlug Russell\cite{basic_emotions_theories} eine zweidimensionale Darstellung mit sechs primären Emotionen nach Ekam vor (siehe Tabelle \ref{vergleich-basisemotionen}). 
Emotionen werden also dank dieses Modells (siehe Abbildung \ref{russell}) eine horizontale Komponente haben: Valenz (Freude/Verdruss) und eine vertikale Komponente: Erregung (Aktivierung). 


\begin{figure}[h] \centering
\includegraphics[width=12cm]{Images/russell.png} 
\vspace{-0.3cm} 
\caption[Einordnung von emotional Erfahrung nach Russell.]{ Einordnung von emotional Erfahrung nach Russell\cite{basic_emotions_theories}. }
\label{russell} 
\end{figure}


Die Valenz unterscheidet positive Emotionen von negative Emotionen, und die Erregung informiert über körperliche Erregung, die man durch die Anzahl von physiologischen Reaktion feststellen kann. 
Jede Emotion lässt sich als Kombination dieser beiden Parameter darstellen was für eine mathematische auswertung sehr hilfreich sein könnte. 
Dieses Modell hat viel Erfolg gehabt, da er die Darstellung von eine Unendlichkeit von Emotionen erlaubt hat. 
Trotz der unterschiedlichen Anzahl von Dimensionen von verschiedenen Autoren vorgeschlagen, die beiden Dimensionen von Russell (Valenz und Erregung) sind Faktoren die in fast alle Modelle dieses Ansatzes auftreten.





% Unterkapitel
\subsection{Virtual Reality (VR)} \label{grund-vr}


\todo[inline]{Verantwortlich: Arnaud, Boris}

% Unterkapitel
\subsection{Sensoren und biophysiologische Signale zur Emotionserkennung} \label{grund-sensoren-subsec}

\todo[inline]{Verantwortlich: Arnaud, Kevin}

% Unterkapitel
\subsubsection{Körpertemperatur-Sensor} \label{grund-temp-subsubsec}

\todo[inline,color=green!40]{RfP}

Die Temperatur ist die nach der Zeit am zweithäufigsten gemessene physikalische Größe.
Um Temperaturen messen zu können, benötigt man einen Sensor, der die Temperatur in eine
Spannung oder einen Strom umsetzt. Zudem benötigt man einen Analog-Digital-Wandler
(deutsch: AD-Wandler, englisch: analog-digital-converter = ADC), der das analoge Signal digitalisiert. Diese sind entweder direkt auf dem Sensor angebracht oder in dem Mikrocontroller
integriert. Natürlich gibt es eine Reihe von verschiedenen Temperatursensoren, die mit temperaturabhängigem Widerstand bis zum fertigen All-in-one-Bauteil mit digitalem Ausgang
ausgestattet sind. Mit dem Temperatursensor kann die Körperschalentemperatur gemessen werden, die sich je nach Gemütslage leicht verändern kann. Die Körperschale beinhaltet die stoffwechselarmen, peripheren Anteile der Extremitäten und die gesamte Haut. Von der Körperschalentemperatur wird die Körperkerntemperatur abgegrenzt. Das heißt, dass die Körperschalentemperatur sehr in Abhängigkeit von der Umgebungstemperatur schwankt und durch Konvektion und Konduktion von der Körperkerntemperatur beeinflusst wird.  Bei allen Sensoren sollte man auf die jeweilige Auflösung und Genauigkeit achten um zuverlässige Werte für die Emotionsforschung zu erhalten!

% Unterkapitel
\subsubsection{Blood Volume Pulse-Sensor (BVP)} \label{grund-bvp-subsubsec}


Mit einem BVP-Sensor wird der Blutvolumen-Puls gemessen, der mit der Herzrate in Korrelation
steht und somit als ein Zeichen für den Erregungsgrad fungiert. Der Sensor besitzt
eine rote oder grüne Lichtquelle (Emitter) und einen optischen, nicht-invasiven Sensor, der
die kardiovaskulären Änderungen durch unterschiedliche Lichtdurchlässigkeiten, auch Transluzenz
genannt, in der Arterie misst. 

Auch hier gibt es verschiedene Ansätze zum Messen des Blutvolumen-Pulses. Entweder wird die Lichtdurchlässigkeit einer Extremität (z.B. Finger oder Ohrläppchen) oder die Reflektion des Lichts in der Arterie gemessen. 

Wenn das Herz Blut durch die Arterien pumpt, wird die Lichtdurchlässigkeit undurchsichtiger und weniger Licht (durch eine größere Lichtabsorbtion) passiert die Arterie vom Licht-Emitter zum Sensor. Beim zweiten Verfahren wird mehr Licht durch die Arterie reflektiert
und trifft auf den Sensor.

Die ermittelten Werte sagen etwas über die Sensoren und biophysiologische Signale zur Emotionserkennung Gefäßsystem aus, beispielsweise ob Verkrampfungen oder Verspannungen vorliegen. Ebenfalls kann durch eine kleine Pulsamplitude auf eine Migräne geschlossen werden.

% Unterkapitel
\subsubsection{Messen der Sauerstoffs{\"a}ttigung (SpO2)} \label{grund-spo2-subsubsec}




% Unterkapitel
\subsubsection{Galvanic Skin Response (GSR)} \label{grund-gsr-subsubsec}


Galvanic Skin Repsonse(GSR) bezeichnet ein kurzzeitige Änderung der elektrischen Hautleitfähigkeit, und wird auch gebräuchlicher weiße als elektrodermale Aktivität (EDA) bezeichnet. So können emotionale Reaktionen unbewusst zu einer erhöhten Schweißsekretion führen, wodurch die Hautleitfähigkeit für eine kurze Zeit erhöht wird. Da die Haut nicht der kognitiven Kontrolle unterliegt, sondern dem vegetativen Nervensystem, lassen sich mit Hilfe von GSR physiologische und psychologische Prozesse im Körper wahrnehmen und messen.  Die Haut ist funktionell das vielseitigste Organ des menschlichen Körpers und Schnittstelle zur Umgebung. Zusammen mit anderen Organen bildet sie für unseren Körper einen Schutz vor Umwelteinflüssen, besitzt eine Temperaturregulierung und verleiht dem Körper einen Tastsinn durch eine Vielzahl von unterschiedlichen Rezeptoren.  Der Körper ist mit über drei Millionen Schweißdrüsen übersät, die meisten befinden sich dabei auf Stirn, Fußsohlen, Händen und unter den Armen. Aus der Annahme, Lügen verstärkt eine sympatische Erregung, resultiert die Schlussfolgerung, dass sich das Verfahren GSR auch als Indikator für Falschaussagen beim Lügendetektor verwenden ließe. So entsteht die Frage, ob durch Videos, Bilder, Töne oder andere sensorische Anreize Emotionen stimuliert und gemessen werden können.


% Unterkapitel
\subsubsection{Elektroenzephalografie (EEG)} \label{grund-eeg-subsubsec}




% Unterkapitel
\subsubsection{Elektrookulografie (EOG)} \label{grund-eog-subsubsec}


Die Elektrookulografie ist ein Messverfahren zur Ermittlung des Netzhautruhepotenzials und wird häufig in der Diagnostik zur Ermittlung von Erkrankungen, z.B. von Gleichgewichtsstörungen, Nystagmus (unkontrollierte rhythmische Augenbewegungen) oder Erkrankungen der Netzhaut, verwendet. Das Messverfahren wird auch Elektronystagmographie genannt. Bei der EOG-Messung  wird entweder die Bewegung der Augen, oder aber die Veränderungen des Ruhepotentials der Netzhaut gemessen. Mit dem Ruhepotenzial der Netzhaut ist eine permanent bestehende Spannungsdifferenz zwischen der Rückseite und der Vorderseite des Augapfels gemeint. Um das Ruhepotenzial zu messen, werden zwei Elektroden benötigt, die jeweils paarweise entweder rechts und links (in diesem Fall beider Augen) oder aber oberhalb und unterhalb eines Auges angebracht werden. Eine dritte Elektrode dient dabei lediglich als Referenz zur Messung der Differenzen. Dadurch lassen sich kleinste Augenbewegungen ermitteln. Das Ruhepotenzial verändert sich bei einer Augenbewegung, da die Vorderseite des Auges näher an eine der Elektroden heran kommt. Die Rückseite des Auges nähert sich dagegen der gegenüberliegenden Elektrode. Die Elektrookulografie wird in aller Regel ambulant durchgeführt und ist für den Patienten mit keinerlei Schmerzen, Risiken Sensoren und biophysiologische Signale zur Emotionserkennung oder Nebenwirkungen verbunden. Durch dieses Messverfahren können ebenfalls Blickbewegungsmessungen aufgezeichnet werden, wodurch der Blickverlauf einer Person sichtbar gemacht werden kann. Diese Methode wird als Eye-Tracking bezeichnet und ist eine experimentelle Methode zur Gewinnung von Emotionen und Gehirnaktivitäten. Durch Kombination weiterer biometrischer Sensoren, kann das menschliche Verhalten in Situationen in Virtual-Reality differenzierter analysiert und ausgewertet werden und könnte den Lern- und Bildungseffekt während Schulungen weiter fördern.

% Unterkapitel
\subsubsection{Analog/Digital-Wandler} \label{grund-ad-wandler-subsubsec}




% Unterkapitel
\subsection{Kommunikation} \label{grund-kommunikation}

\todo[inline]{Verantwortlich: Kevin, Jonas}

Zur Übertragung der gewonnenen Sensordaten, wie auch zur Kommunikation der einzelnen Teilkomponeneten untereinander, wurden verschiedene Technologien und Protokolle evaluiert. \\

Die Implementierung erfolgte schlussendlich mit UDP Broadcasts und einem rudimentären Kommunikationsprotokoll.
Für die Übertragung war es wichtig, dass die Implementierung sowohl auf Seiten der Unreal Engine, als auch beim Mikrocontroller und bei der Kontrollanwendung einfach und mit geringem Arbeitsaufwand möglich ist.
Zugleich sollte auf eine einfache Wartbarkeit und die Möglichkeit die Verbindung ``live'' zu debuggen geachtet werden. \\

Die ersten Tests erfolgten mit einer einfachen Ausgabe der Sensordaten über eine USB-Serial Verbindung.
Da jedoch der Sensor abgesetzt und drahtlos Daten senden sollte, wurde dann beim ESP32 UDP Broadcasts als Kommunikationsweg gewählt. Neben der einfachen Implementierung auf allen Teilkomponenten ist hierbei auch die Möglichkeit einer 1:n Kommunikation gegeben.
In Zukunft sollte darüber nachgedacht werden, Broadcasts durch Multicasts zu ersetzen, da dadurch in einem n:m Szenario eine einfachere Separierung der Datenströme der einzelnen Sensoren erfolgen kann. \\

Jeder Teilnehmer wird durch eine einzigartige ID identifiziert, die in jedem Paket mitgesendet wird. \\

(BILD EINFÜGEN UNTERSCHIEDLICHE DATENPAKETE) \\

Der Sensor sendet hierbei nicht für jeden Datenwert ein Paket, sondern aggregiert immer 75 Datenwerte in ein Paket. Dabei werden die einzelnen Kanäle in 4 Pakete aufgeteilt.
Paket 1 enthält die Sensordaten für Temperatur, GSR und EEG1. 
Paket 2 die für. 
Paket 3 für. 
Paket 4 zuletzt dann die Daten für IR-RAW. Diese Aggregierung und Aufteilung wurde durch Instabilitäten in Sensor Netzwerkstack erforderlich.
Trotzdem ist durch diese Lösung weiterhin eine Datenrate von 250 Samples/Sekunde möglich. Diese liegt weit über dem theoretischen Minimum, was für eine verlustfreie Rekonstruktion des Signals nötig wäre, so dass auch weiterhin die Kommunikation robust gegenüber Störeinflüssen ist.




% Unterkapitel


% Unterkapitel
\subsection{Grundlagen der Mustererkennung} \label{grundlagen-mustererkennung-subsec}

\todo[inline,color=green!40]{Verantwortlich: Artur \\
- Bereit zum Korrekturlesen.}

Mustererkennung (enlg. ``pattern recognition") ist ein Unterthema des machinellen Lernens.
Das Ziel besteht darin, automatisierte Systeme zu entwerfen, die hoch abstrakte Muster in Daten erkennen können.
Konkret heißt dies, dass man Maschinen beibringen möchte komplexer Aufgaben zu lösen, welche vom Menschen nahzu mühlelos und natürlich erledigt werden können.
Typische Beispiele für die zahlreichen Anwendungsbereiche sind die Objekterkennung, Spracherkennung sowie die Erkennung und Verfolgung in Bildern. 
Die Emotionserkennung ist ein Anwendungsbereich der Mustererkennung.
Die Hauptidee hinter der Lösung eines Mustererkennung-Problems ist es, dieses als Klassifikationsproblem zu übersetzen, wobei die zu erkennende Mustern die unterschiedliche Klassen bilden. 
Die vom Mustererkennungs-System eingegebenen Daten werden dann verarbeitet und der ``am nächsten liegenden" Klasse zugeordnet.
Beispielsweise können bei der Emotionserkennung die Eingangsdaten Bilder oder physiologische Signale sein, die in verschiedene Klassen eingeteilt werden, welche jeweils einer Emotion entsprechen. \\

Ein wichtiger Teil eines jeden Mustererkennung-Problems ist der Lernansatz, mit welchem die Maschine lernen soll die Muster in den Daten zu erkennen. 
Traditionell werden zwei Ansätze verwendet:

\begin{itemize}%[noitemsep]
  \item \underline{Überwachter Lernansatz:}
  Dieser Ansatz kann nur verwendet werden, wenn vor der Verarbeitung der Daten ein Datenbeschriftungsschritt durchgeführt wurde.
  In diesem Schritt wird jedem Element des Datensatzes ein Etikett (engl. ``label") zugewiesen, das angibt, welcher Klasse der jeweilige Datenpunkt zugeordnet werden kann.
  Die zusätzlichen Informationen, die die Etiketten liefern, werden als Grundlage verwendet, um sie mit der Vorhersage des Systems zu vergleichen und zu korrigieren, wenn sie nicht gleich sind.

  \item \underline{Unüberwachter Lernansatz:}
  Dieser Ansatz wird verwendet, wenn keine Etiketten für die Daten vorhanden sind.
  Unüberwachte Lerntechniken zielen darauf ab, der Maschine beizubringen, Muster in den Daten selbst zu finden. 
  Sie werden meist verwendet, um Einblicke in Daten zu erhalten, deren Struktur unbekannt ist.
\end{itemize} %\vspace{0.5cm}


Überwachtes Lernen liefert aktuell weit bessere Ergebnisse, jedoch ist die Beschriftung mit Etiketten der Daten nicht immer einfach oder teilweise sogar gar nicht möglich (z.B. wenn die Datenmenge sehr groß ist oder wenn Unsicherheit über die Vergabe der Etikette besteht).
Aus diesem Grund wächst das Interesse an unüberwachter Lernansätzen.
Diese Ansätze sind jedoch schwierig zu benutzen, da sie eine große Menge an Daten voraussetzen.
Kompromisse sind mit semi-überwachten Lernansätzen möglich, bei denen die Daten für einen Teil des Datensatzes (aber nicht für den ganzen Datensatz) mit Etiketten beschriftet und damit bekannt sind.
In diesem Fall kann eine Mischung aus überwachten und Unüberwachter Techniken angewendet werden \cite{Zhu2008}. \\


Im Rahmen des ELISE-Projekts werden mit Hilfe von Mustererkennungsverfahren eindimensionale Zeitsignale von physiologischen Sensoren in Echtzeit für die Erkennung von drei Emotionen verarbeitet: Glück, Frustation und Langeweile. Um den Emotionsklassifizierer aufzubauen, wird ein standardmäßiger, überwachter Lernansatz namens Emotion Recognition Chain verwendet, der im folgendem Kapitel beschrieben wird. \\


% Unterkapitel 
\subsection{Emotion Recognition Chain} \label{emotion-recogniton-chain-subsec}

\todo[inline]{Verantwortlich: Artur \\
- ERC Bild ändern (5 statt 4 Schritte)}

Die Emotion Recognition Chain (ERC) besteht aus fünf Hauptschritten: 
Datenerfassung, Vorverarbeitung, Segmentierung, Merkmalsextraktion und Klassifizierung (vgl. Abb. \ref{fig:erc}).
In den folgenden Unterkapiteln wird für jeden Schritt eine allgemeine Erklärung gegeben. \\


\begin{figure}[h]
\includegraphics[width=\textwidth]{Images/erc.png} 
\vspace{-0.3cm} 
\caption[Emotion Recognition Chain]{Emotion Recognition Chain: Zeitreihen-Datensätze werden von tragbaren Sensoren aufgenommen (Datenerfassung) und vorverarbeitet (Vorverarbeitung). Die Daten werden dann in Segmente unterteilt (Segmentierung), aus denen Merkmale extrahiert werden (Merkmalsextraktion). Mit den gewonnenen Merkmalen wird schließlich ein Klassifikator trainiert und anschließend dessen Ergebnisse bewertet (Klassifikation).}
\label{fig:erc} 
\end{figure}
%\vspace{0.5cm}


% Unterkapitel
\subsection{Datenerfassung} \label{datenerfassung-4}


In den Kapiteln \ref{datenerfassung-0} und \ref{datenerfassung-1} wurde das verwendete Datenset bereits detailiert beschrieben, sodass hier darauf verzichtet wird. Weitere Inforamtionen zur Hardware, mit der die Daten aufgezeichnet wurden, k{\"o}nnen Kapitel \ref{systementwurf-4} entnommen werden. \\

% Unterkapitel 
\subsubsection{Vorverarbeitung} \label{vorverarbeitung-0}

Das Ziel der Vorverarbeitung ist die ``Verbesserung'' der Daten f{\"u}r die nachfolgenden Schritte der ERC. In der Regel ist es dadurch besser m{\"o}glich Muster in Daten erkennen zu k{\"o}nnen. 
Vorverarbeite Daten erreicht man durch Anwenudng von z.B. Filterung (Rauschunterdruckung), Normierung oder Reduzierung von unerw{\"u}nschten oder unbedeutenden Datenteilen. \\

% Unterkapitel 
\subsection{Segmentation} \label{segmenation-4}

In diesem Schritt der ERC wurden im Vergleich zum Prototpye 1 (vgl. Kapitel \ref{segmenation-1}) keinerlei Ver{\"a}nderungen durchgef{\"u}hrt. \\

% Unterkapitel
\subsection{Merkmalsextraktion} \label{merkmalsextraftion-1}

Wie bereits in Kapitel \ref{merkmalsextraktion-0} beschrieben, ist das Ziel der Merkmalsextraktion Charakteristiken und Merkmale in den Daten zu finden, die für das
Klassifizierungsproblem von möglichst hoher Relevanz sind. Im Rahmen des ELISE Projektes haben wir verschiedene Vorgehensweisen angewendet. Im den folgenden Unterkapiteln werden diese vorgestellt. \\


% Unterkapitel 
\subsubsection{Handgefertigte Merkmale} \label{hc-features-1}
Der handgefertigten Merkmal Ansatz (enlg. "hand-crafted features approach") besteht in der Berechnung relativ einfacher Merkmale von denen vermudetet wird, dass sie für das Klassifizierungsproblem der Eingangssignale relevant sein können. Diese Vorgehensweise hat den Vorteil des einfachen Aufbaus als auch der relativ geringen benötigten Rechenleistung, wobei potentiell gute Klassifizierungsergebnisse erwarten werden. \\


Obwohl frühere Forschungsarbeiten schon handgefertigte Merkmale zur Emotionserkennung unter mithilfe physiologischer Signale getestet haben (vgl. \cite{martinez_ieee_2013}), wurde dieser Ansatz noch nie für die Erkennung dieser spezifischen Emotionen unter Verwendung dieser Kombination von Sensoren getestet.
Zusätzlich haben wir zuerst handgefertigte Merkmale getestet, um ein Basisergebnis zu liefern, mit der die Ergebnissen der anderen Ansätze vergleichen werden können.
Handgefertigte Merkmale sind in der Regel entweder einfache statistische Werte, Fourier-basierte oder selbstentwickelte Merkmale sein, die aufgrund von Vorkenntnissen der Daten verwendet werden. 
Diese Arbeit wurden statistische, Fourier-basierte und selbstentwickelte Merkmale getestet. \\

\textbf{Statistische Merkmale \\}
Die Tabelle \ref{tab:statistische} fasst die elf verschiedenen und in der Studie verwendeten statistischen Merkmale zusammen \cite{bscpiet}. Wir bezeichnen $\mathbf{x} = (x_1, x_2, ...., x_T) $ als Vektor, der die in einem Datenzeitfenster der Länge $T$ enthaltenen Sensorwerte für einen Sensorkanal darstellt. 


\begin{table}[h]
\begin{tabular}{| l | p{12.5cm} |}
\hline
    \textbf{Merkmalname}     &  \textbf{Definition}  \\ \hline
    
    Mittelwert         & \vspace{0.01cm}
    $ mean(\mathbf{x}) =$ \Large{$\frac{1}{T} \sum_{k=1}^T (x_k) $} \\[0.5cm] \hline 
    
    Standard-Abweichung        & \vspace{0.01cm}
    $ \sigma(\mathbf{x}) =$ \Large{$ \sqrt{ \frac{1}{T} \sum_{k=1}^{T}{(x_k - \mu)^{2}} } $ } \\[0.5cm] \hline
    
    Maximum                   & \vspace{0.01cm}
    $ max(\mathbf{x}) = \max(x_{1},x_{2},\dots ,x_{T}) $
    \\[0.5cm] \hline
    
    Minimum                   & \vspace{0.01cm}
    $ min(\mathbf{x}) = \min(x_{1},x_{2},\dots ,x_{T}) $
    \\[0.5cm] \hline
    
    Amplitude                 & \vspace{0.01cm}
    $ A(\mathbf{x}) = max(\mathbf{x}) - min(\mathbf{x}) $ 
    \\[0.5cm] \hline
    
    25/50/75\% Perzentil      & Wert einer Menge, unter dem 25/50/75\% der Werte aus der Menge fallen. \\ \hline
    
    Interquartiler Bereich    & Differenz zwischen dem 75. und 25. Perzentil.
    \\ \hline
     
    Schräge                   & \vspace{0.01cm}
    $ \gamma _{1}(\mathbf{x}) = \operatorname{E}$ \Large{$\left[\left({\frac {X-\mu }{\sigma }}\right)^{3}\right]$ \normalsize{$=$} ${\frac {\mu _{3}}{\sigma ^{3}}}$ \normalsize{$=$} ${\frac {\operatorname {E} \left[(X-\mu )^{3}\right]}{ (\operatorname {E} \left[(X-\mu )^{2}\right])^{3/2}}}$ \normalsize{$=$} ${\frac {\kappa _{3}}{\kappa _{2}^{3/2}}} $} \vspace{0.2cm}
    \\[0.3cm] \hline
     
    Kurtosis                  & \vspace{0.01cm}
    $ \operatorname {Kurt}[\mathbf{x}] = \operatorname{E} $ \Large{$\left[\left({\frac {X-\mu }{\sigma }}\right)^{4}\right]$ \normalsize{$=$} ${\frac {\mu _{4}}{\sigma ^{4}}}$ \normalsize{$=$} ${\frac {\operatorname {E} [(X-\mu )^{4}]}{(\operatorname {E} [(X-\mu )^{2}])^{2}}} $} \vspace{0.2cm}
    \\[0.3cm] \hline
\end{tabular} 
\caption[Statistische Merkmale]{Statistische Merkmale, die im Rahmen des ELISE-Projektes verwendet wurden. } \label{tab:statistische}
\end{table} 


\textbf{Fourier-basierte Merkmale \\}
Die in diesem Kapitel präsentierten Ergebnisse wurden in einer Bachelorarbeit \cite{bsclittau} ermittelt, die auch ihm Rahmen des ELISE-Projektes geschrieben wurde. \\

Die Transformation von Zeitsignalen im Frequenzbereich bei Studien zu einem gängigen Werkzeug für die Analyse von Zeitreihendaten geworden. Es basiert auf einem mathematischen Prozess namens Fouriertransform, wo ein Signal in einer unendlichen gewichteten Summe von Sinus- und Coisnewellen unterschiedlicher Frequenz zu zerlegen.
Bei einer kontinuierlichen Funktion $ f : t \rightarrow f(t) $ ist der Fouriertransformation $ F(w) $ eines Signals $ f $: 
\begin{equation} 
\Large{ F(w) = \int\limits_{-\infty}^{\infty}{f(t)e^{-2\pi{}iw}dt}} 
\label{equ:fourier} \end{equation}
\vspace{0.2cm}

In den meisten realen Anwendungen sind die Signale jedoch diskontinuierlich. 
Eine alternative Fourier-Transformation, die sogenannte diskrete Fourier-Transformation (engl. "discrete fourier transform"), kann dann stattdessen angewendet werden.
Ein diskretes Signal $ x = (x_{0},x_{1},x_{2},...,x_{N-1}) $ von N Punkten wird durch die folgende Formel gegeben:
\begin{equation} 
\Large{ X_{k} = \sum\limits_{n=0}^{N-1} x_{n} e^{- \frac{2 \pi i}{N} kn} } 
\end{equation} 
\vspace{0.2cm}

wobei die Koeffizienten $ X_{k} \in \mathbb{C} $ die Fourier-Koeffizienten des Originalsignals sind.
Die Fourier-Koeffizienten werden hauptsächlich verwendet, um das Leistungsspektrum (engl. "power spectrum") $ p(x) $ des Signals $x$ zu berechnen, das Auskunft über den Beitrag jeder Frequenzkomponente zum Signal gibt. 
Die Komponenten $ p_{k}(x) $ des Leistungsspektrums für $ k \in \lbrace 0,1,...,N-1 \rbrace$ werden berechnet durch: 
\begin{equation} 
\Large{ p_{k}(x) = \Re(X_{k}^{2})+\Im(X_{k}^{2}) } 
\end{equation} 

wobei $ p_{k}(x) $ mit der Energie des Singals $x$ verglichen werden kann, das der $k$-ten Frequenz zugeordnet ist. \\


Um frequenzbezogene Merkmale zu extrahieren, wird das Leistungsspektrum der Signale in vier Frequenzbänder gleicher Länge unterteilt.
Der Mittelwert, die Standard-Abweichung, das Maximum und das Minimum wurde dann auf jedem Frequenzband für jeden Sensorkanal extrahiert, was zu einem Prozess der Merkmalsextraktion aus normalisierten Datenrahmen führte, wie in Abbildung \ref{fig:fft} dargestellt. 

\begin{figure}[h] \centering{
\includegraphics[width=11cm]{Images/ttf.png}}
\caption[Merkmalsextraktion aus frequenzbezogener Domain]{ Merkmalsextraktion aus frequenzbezogener Domain. } 
\label{fig:fft} \end{figure} \vspace{0.5cm}


Zusätzlich wurde für jeden Sensorkanal des gesamten Spektrums die Standard-Abweichung  extrahiert. Dies führt zu insgesamt $9 \ast (1 + 4 \ast 4) = 153$ Merkmalen im Frequenzbereich pro Zeitfenster. \\


\textbf{Selbstentwickelte Merkmale \\}
Es wurden zwei eigene Merkmale definiert \cite{bscpiet}. Nulldurchgang (engl. "zero crossing") und Anzahl der Spitzen (engl. "number of peaks"). Im Folgendem werden diese beiden Merkmale detailiert beschrieben. \\

Das Nulldurchgang-Merkmal zählt die Häufigkeit, mit der das Signal eines Sensorkanals in einem Zeitfenster die Nulllinie überschreitet.
Alle Sensorsignale wurden durch Normierung verarbeitet (vgl. Kapitel \ref{vorverarbeitung-1}) und damit wurden alle Mittelwerte auf Null zentriert.
Um zu vermeiden, dass Rauschen entlang der Nulllinie in dem Merkmal gezählt wird, wird nur ein Nulldurchgang in einer bestimmten Zeitspanne gezählt. \\


Das Spitzenzähler-Merkmal bestimmt die Anzahl von loklanen Hochpunkten im Zeitsignal.
Alle lokalen Maximen sind durch einen Onset (Startpunkt), eine Spitze und einen Offset (Endpunkt) gekennzeichnet (vgl. Abbildung \ref{fig:peaks}). 
Jedes Vorkommen einer Onset/Offset-Paarung wird hierbei als Spitze gezählt.
Onsets, Spitzen und Offsets werden durch die folgenden Operationen identifiziert (vgl. \cite{bscGouverneur}):

\begin{itemize} %[noitemsep]
  \item Ein Onset wird bestimmt, wenn der Wert des Signals an diesem Punkt nicht negativ ist und die Differenz zwischen ihm und dem nächsten größer als ein vordefinierter Schwellenwert (engl. "threshold") ist.

  \item Ein Offset wird bestimmt, wenn der Wert des Signals kleiner als der Wert des zuletzt gesetzten Onsets ist.

  \item Das lokale Maximum zwischen einem Onset und Offset wird als Spitze bezeichnet.
\end{itemize} \vspace{0.2cm}


\begin{figure}[h] \centering{
\includegraphics[width=11cm]{Images/peaks.png}}
\caption[Spitzenzähler-Merkmal]{Spitzenzähler-Merkmal: Onset (Startpunkt), Spitze und Offset (Endpunkt). Jedes Paar von Onset/Offset erhöht die Anzahl der Spitzen um eins.} 
\label{fig:peaks} \end{figure} \vspace{0.5cm}


Jedes handgefertigte Merkmal wird auf einem Zeitfenster von Daten für jeden Sensorkanal unabhängig voneinander angewendet. 
Jedes Zeitfenster ist daher 117 Merkmalen zugeordnet (9 Sensorkanäle multipliziert mit 13 Merkmalen). \\





% Unterkapitel 
\subsubsection{Codebook Approach} \label{ca-1}


Die bisherige Methodik mit handgefertigten Merkmalen ist klassisch für den überwachten Lernansatz. 
Es existieren aber auch einige Nachteile. 
Das Hauptproblem besteht darin, dass nicht sichergestellt werden kann, dass die gewählten Merkmale die besten Klassifizierungsergebnisse erzielen. Damit besteht immer die Gefahr, dass möglicherweise andere Merkmalen bessere Ergebnisse liefern würden, diese handgefertigten Merkmale aber die  nicht gefunden wurden. Dieses Risiko besteht insbesondere bei der physiologischen Signalverarbeitung zur Emotionserkennung, wo die Struktur der Daten noch recht unbekannt und allgemein komplex ist. 
Eine weitere Schwierigkeit besteht darin, relevante selbstentwickelte Features ohne Expertenwissen über die Daten zu finden.
Darüber hinaus wurden noch keine gut funktionierenden State-of-the-Art handgefertigten Merkmale identifiziert.
Aus diesen Gründen ist es interessant halbautomatische und unüberwachter Ansätze der Merkmalsextraktion zu verwenden und zu testen. \\


K. Shirahama et al. \cite{kimiaki_codebook_approach_2016} schlugen eine unüberwachte Merkmalsextraktionsmethode namens Codebook Approach (CA) vor, um Merkmale aus 1D-Zeitreihensignalen zu erzeugen.
Der CA hat den Vorteil, dass formbasierte Merkmale gefunden werden können, die für das Problem der Emotionserkennung relevant sind, aber weder offensichtlich noch leicht als Mensch zu interpretieren sind. 
Der CA besteht aus drei Schritten, die in den folgenden Abschnitten erläutert werden: Codebuchkonstruktion (engl. "codebook construction"), Codewortzuordnung (engl. "codeword assignment") und der anschließenden Klassifizierung. \\


\textbf{Codebuchkonstruktion \\}
Ziel dieses Schrittes ist es, Teilsequenzen (sogenannte "Codewörter") zu bestimmen, die für die 1D-Eingangssensorik charakteristisch sind. 
Dies wird erreicht, indem Zeitfenster aus dem ursprünglichen Datensatz für jeden Sensorkanal unabhängig voneinander nach dem im Kapitel \ref{segmenation-1} definierten Segmentierungsansatz extrahiert werden.
Aus jedem so erhaltenen Zeitfenster der Größe $T$ werden kleinere Segmente der Größe $\alpha$ unterteilt.
Ein Clustering-Algorithmus wird dann auf die Menge der Segmente $\alpha$ angewendet, um Clusterzentren zu finden.
Nach der Konvergenz werden die Clusterzentren als Codewörter betrachtet und zum Aufbau einer Sammlung von Codewörtern mit dem Namen ``Codebuch'' verwendet, wie in Abbildung \ref{fig:ca_construction} aus \cite{kimiaki_codebook_approach_2016} dargestellt. 
Die Anzahl der Codewörter (d.h. die Größe des Codebuchs oder die Anzahl der Cluster) ist ein Hyperparameter des Verfahrens. Im Rahmen dieser Arbeit wurde ein k-means Clustering-Algorithmus verwendet, um die Codewörter auf den ELISE-Daten zu erhalten. \\


\begin{figure}[h]\centering{
\includegraphics[width=\textwidth]{Images/CA_construction.png}}
\caption[Codebuchkonstruktion]{Codebuchkonstruktion: Zeitfenster von Daten werden zunächst extrahiert. Ein Clustering-Algorithmus wird dann auf das alle Zeitfenster angewendet, um Clusterzentren (und damit Codewörter) zu finden, die zum Aufbau des Codebuchs verwendet werden. }
\label{fig:ca_construction} \end{figure} \vspace{0.5cm}


\textbf{Codewortzuordnung \\}
Nach der Konstruktion der Codewörter wird für jedes Zeitfenster $T$ ein histogrammbasierter Merkmalsvektor erstellt (vgl. Abbildung \ref{fig:ca_assignment}, entnommen aus \cite{kimiaki_codebook_approach_2016}). 
Der zu klassifizierende Datensatz wird zunächst in Zeitfenster der Größe $T$ segmentiert, aus denen Segmente der Größe $\alpha$ nach dem gleichen Verfahren wie beim Aufbau des Codebuchs extrahiert werden. 
Jedes Segment $\alpha$ wird dann mit den Codewörtern verglichen, so dass das "ähnlichste" Codewort gefunden werden kann. 
Ein K-Bin-Histogramm (mit $K$ Anzahl der Codewörter) mit Informationen über die Anzahl der Male enthält, die jedes Codewort als am ähnlichsten zu den Segmenten $\alpha$ im Zeitfenster $T$ betrachtet wurde, wird dann erstellt und als Merkmalsvektor verwendet, um das zu klassifizierende Zeitfenster $T$ darzustellen. 
Das Maß für die Ähnlichkeit von Codewörtern und Datensegmenten $\alpha$ basiert auf der euklidischen Entfernung. \\


\begin{figure}[h]\centering{
\includegraphics[width=\textwidth]{Images/CA_assignment.png}}
\caption[Codewortzuweisung]{Codewortzuweisung: Jedes Datensegment wird mit den Codewörtern verglichen und ein Histogramm mit Informationen darüber, wie oft jedes Codewort als "am ähnlichsten" betrachtet wurde, wird erstellt. }
\label{fig:ca_assignment} \end{figure} \vspace{0.5cm}


Der zuvor beschriebene Ansatz wird als "Hard-Zuordnung" (engl. "hard assignment") bezeichnet, da Datensegmente einem einzigen Codewort zuzuordnen werden.
Ein Nachteil dieser Vorgehensweise ist die mangelnde Flexibilität im Umgang mit potenzieller Unsicherheit bei der Codewortzuweisung, die z.B. auftreten kann, wenn ein Zeitfenster von Daten zwei oder mehr Codewörtern sehr ähnlich ist, da es nur einem zugeordnet werden kann. 
Ein alternativer Ansatz dieses Problem zu umgehen ist die so genannte "Soft-Zuordnung" (engl. "soft assignment"). Hierbei werden auch alle Codewörter des Codebuchs Datensegmenten zuzuordnen, wobei die Ähnlichkeit jeweils als ein Bin im Histogramms dargestellt wird. Sehr ähnlich entspricht hierbei einem hohen Wert und nicht ähnlich einem kleinen Wert (anstatt 0 oder 1 wie in der Variante der Hard-Zuordnung). 
Eine Kerneldichtefunktion (vgl. \cite{gemert_ieee_2009}) wird verwedent, um die Histogramm-Bins zu berechnen. Es wurde die folgende Funktion verwendet:


\begin{equation} 
\Large{ {f(\alpha,c_k,p) = \frac{1}{\gamma}} \frac{g(\alpha,c_k,p)}{\sum_{i=1}^{K} g(\alpha,c_i,p)}}
\end{equation}
\newline
wobei $f(\alpha,c_k,p)$ die Ähnlichkeit des Segments $\alpha$ bezeichnet, das sich auf das Codewort $c_k$ bezieht. $p$ ist ein Glättungsparameter (ein großes $p$ bewirkt eine starke Glättung), $\gamma$ ist die Anzahl der Segmente im Zeitfenster $T$ und: \\
\begin{equation} 
\Large{ { g(\alpha,c_k,p) = \frac{1}{\sqrt{2 \pi p^{ 2 }}}} exp(\frac{d(\alpha,c_k)}{2p^{2}}). }
\label{equ:codeword_assignment_2} \end{equation}
\newline
Um numerischen Unterlauf zu vermeiden, wird die Gleichung (\ref{equ:codeword_assignment_2}) zunächst mit dem Log-Sum-Exp-Trick (vgl. \cite{murphy_2012}) berechnet. \\


\textbf{Fusion mehrerer Sensoren \\}
Oftmals werden mehrere Sensoren verwendet, die gleichzeitig mehrere verschiedene Signale derselben Emotion erzeugen.
Die Fusion dieser Signale ist wichtig, da sie die Genauigkeit der Emotionserkennung verbessern kann.
Es gibt zwei verschiedene Ansätze (vgl. \cite{snoek_2005}): die frühe Fusion (engl. "early fusion") und die späte Fusion (engl. "late fusion"): 

\begin{itemize}
  \item \underline{Frühe Fusion:} In der Dimension $K \times S$ wird nur ein Klassifizierer benötigt, wobei $K$ die Anzahl der Codewörter und $S$ die Anzahl der Sensorkanäle ist. Der Klassifikator wird anhand der Verkettung von Codebuchmerkmalen trainiert und ausgewertet, die auf jedem Sensorkanal unabhängig voneinander berechnet wurden.

  \item \underline{Späte Fusion:} Erfordert mindestens $S$-Klassifikatoren (einen für jeden Sensorkanal). Ein Klassifizierer wird unabhängig für jeden Sensorkanal unter Verwendung der für den betrachteten Sensor erhaltenen Codebuchmerkmale trainiert. Die Vorhersagen der $S$-Klassifikatoren werden dann fusioniert, um die Klassenbezeichnung des zu klassifizierenden Zeitfensters zu schätzen (z.B. mit einem zusätzlichen Klassifizierer).
\end{itemize} \vspace{0.5cm}


In dem ELISE Projekt verwenden wir den späten Fusionsansatz, weil er rechnerisch günstiger ist und von K. Shirahama (vgl. \cite{kimiaki_codebook_approach_2016}) empfohlen wird. \\



% Unterkapitel 
\subsubsection{Merkmals Auswahl} \label{featureSelection-1}

Um bessere Kombinationen von handgefertigten Merkmalen zu finden, kann ein Bottom-Up-Merkmal-Auswahlalgorithmus verwendet werden, der auf der Prüfung von Gruppen von Merkmalen mit zunehmender Größe basiert.
Zunächst wird das Merkmal mit der höchsten Klassifizierungsperformance unter allen verfügbaren Merkmalen ausgewählt. 
Anschließend wird die Performance von Gruppen von zwei Merkmalen, die sich aus dem ausgewählten Merkmal und der Reihe nach jedem anderem Merkmal zusammensetzen, werden dann getestet und das beste Paar wird ausgewählt. 
Dieser Prozess wird so lange wiederholt wiederholt, bis alle relevanten Merkmale verwendet wurden. 
Am Ende wird die beste Kombination von Merkmalen ausgewählt.
Es sei darauf hingewiesen, dass es sich um einen heuristischen Algorithmus handelt, so dass es möglich ist, dass unter Umständen die absolut beste Kombination nicht zwingend gefunden wird. \\

Der folgende Pseudocode beschreibt unseren Algorithmus zur Merkmalsauswahl: \\

\begin{algorithm}[H]
    \SetKwInOut{Input}{Input}
    \SetKwInOut{Output}{Output}
    
    %\vspace{0.1cm}

    %\underline{function feature-selection} $(features, C, kernel, \gamma)$\;
    
    \textbf{Input parameters:}  \\
    \hspace{0.5cm} - $C$, $\gamma$ = C-SVM params \\
    \hspace{0.5cm} - $candidates$ = $[f1,f2,...,fn]$ list of n features to test \\
    \hspace{0.5cm} - $training\_set$ = set of all features computed on the training data \\
    \hspace{0.5cm} - $testing\_set$ = set of all features computed on the testing data 
    %\vspace{0.1cm}
    
    \textbf{Output parameters:} \\
    \hspace{0.5cm} - $best\_feature\_combination$ = list of the best features \\
    \hspace{0.5cm} - $best\_accuracy$ = classification accuracy obtained by the best features \\
    \hspace{0.5cm} - $feature\_ranking$ = list ranking features in decreasing order of relevance
    %\vspace{0.1cm}
    
    \hrulefill 
    
    %\vspace{0.1cm}
    %\textbf{Pseudo-code:} \\
    \textbf{Begin} \\
    $candidates\_to\_test$ = $candidates$ \\
    $current\_best\_features$ = $\emptyset$ \\
    $current\_best\_accuracy$ = -1 \\
    $all\_time\_best\_features$ = $\emptyset$ \\ 
    $all\_time\_best\_accuracy$ = -1 \\
    $feature\_ranking$ = $\emptyset$ 

     \While{$candidates\_to\_test \neq \emptyset$}{
         \For{$feature f$ \textbf{in} $candidates\_to\_test$}{
             $trained\_svm$ = $train\_svm(C, \gamma, training\_set, current\_best\_features \cup \{f\}$) \\
             $accuracy$ = $evaluate\_svm(trained\_svm, testing\_set, current\_best\_features \cup \{f\}$)
        
             \If{$accuracy > current\_best\_accuracy$}{
                 $best\_feature\_of\_iteration$ = $f$ \\
                 $current\_best\_accuracy$ = $accuracy$
            }
                
             \If{$accuracy > all\_time\_best\_accuracy$}{
                 $all\_time\_best\_features$ = $current\_best\_features \cup \{f\}$ \\
                $all\_time\_best\_accuracy$ = $accuracy$ 
            }
        }
         $feature\_ranking$ = $feature\_ranking \cup [best\_feature\_of\_iteration]$ \\
         $current\_best\_features$ = $current\_best\_features \cup [best\_feature\_of\_iteration]$ \\
        $candidates\_to\_test$ = $candidates\_to\_test \setminus [best\_feature\_of\_iteration]$ \\
    }
     \textbf{return} $all\_time\_best\_features, all\_time\_best\_accuracy, feature\_ranking$
    

    %\vspace{0.3cm}
    \caption{ Merkmalsauswahl-Algorithmus. }
\end{algorithm}
\vspace{0.5cm}

% Unterkapitel
\subsection{Klassifikation} \label{klassifikation-4}

\todo[inline]{Verantwortlich: Artur \\}



