\subsection{Gliederung dieser Dokumentation} \label{gliederung-subsec}


Die Dokumentation ist in mehreren Teil gegliedert: eine Einführung, die Entwicklung des ersten Prototypen, die Entwicklung des zweiten Prototypen, die Entwicklung des dritten/finalen Prototypen und ein Schlussteil. \\

Einführung: \\

Die Dokumentation fängt mit einer Einleitung an, die den Hintergrund und die Motivation, die Elise Projektbeschreibung, die Gliederung dieser Dokumentation und den mit der Dokumentation übergebenen Anhang erläutert (geschrieben von Minas Michail). 
Der nächste Punkt ``Organisation'' beinhaltet zum einen die Verantwortungsbereiche der aufgeteilten Aufgaben innerhalb der Gruppe und  zum anderen wird erläutert wie und wann die Gruppentreffen stattgefunden haben (geschrieben von Artur Piet). 
Daraufhin werden die Grundlagen übermittelt, um späteres geschehen in der Dokumentation besser nach voll ziehen zu können. 
Der Punkt Grundlagen enthält die Definition von Emotionen (geschrieben von Arnaud Eric Toham Waffo), die Grundlagen der verwendeten Hardware und Software (geschrieben von Arnaud Eric Toham Waffo), die Grundlagen der verwendeten Sensoren und biophysiologischen Signale (geschrieben von Kevin Orth), die Grundlagen der Kommunikation zwischen den gewonnen Sensordaten und dem Board (geschrieben von Kevin Orth \& Jonas Pöhler), die Grundlagen der Mustererkennung (geschrieben von Artur Piet) und die Grundlagen zur Emotionserkennung (geschrieben von Artur Piet). 
Nach den Vermittlungen der Grundlagen \\

Entwicklung des ersten Prototypen: \\

Die Entwicklung des ersten Prototypen umfasst den Systementwurf und das Konzept der Hardware (geschrieben von Kevin Orth), die entwickelte Software zurKommunikation (geschrieben von Kevin Orth \& Jonas Pöhler), die Emotionsinduktion (geschrieben von Meryem Dural, Boris Kamdem \& Minas Michail), die Messreihe (geschrieben von Kevin Orth \& Artur Piet), die Musterkennung (geschrieben von Artur Piet) und die gewonnen Ergebnisse (geschrieben von Artur Piet). \\

Entwicklung des zweiten Prototypen: \\

Die Entwicklung des zweiten Prototypen umfasst den Systementwurf und das Konzept der Hardware und die modifizierte Software zur Kommunikation (beides  geschrieben von Kevin Orth \& Jonas Pöhler). 
Parallel zur Fertigstellung des zweiten Prototypen, lief die Entwicklung der Emotionsinduktion für den dritten Prototypen. \\

Entwicklung des dritten Prototypen: \\

Die Entwicklung des dritten Prototypen umfasst den Systementwurf und das Konzept der Hardware (geschrieben von Kevin Orth), die entwickelte Software zurKommunikation (geschrieben von Kevin Orth \& Jonas Pöhler), die Emotionsinduktion (geschrieben von Meryem Dural, Boris Kamdem \& Minas Michail), die Messreihe (geschrieben von Kevin Orth \& Artur Piet), die Musterkennung (geschrieben von Artur Piet) und die gewonnen Ergebnisse (geschrieben von Artur Piet). \\

Schlussteil: \\

Der Schlussteil beinhaltet das Kapitel Alternative Lösungen aufgeteilt in Kalibrierung (geschrieben von Jonas Pöhler) und Plan B (geschrieben von Meryem Dural) und eine Zusammenfassung (geschrieben von Arnaud Eric Toham Waffo) sowie einen Ausblick (geschrieben von Boris Kamdem).