\subsection{Hintergrund und Motivation} \label{hintergrund-subsec}



Die Gesellschaft befindet sich seit mehreren Jahren in einem beschleunigten Wandel der das Leben ver{\"a}ndert. 
Dieser Wandel wurde durch die Automation und die Digitalisierung, welche sich beide erg{\"a}nzen verst{\"a}rkt und wird auch als vierte industrielle Revolution bezeichnet, die zu Ver{\"a}nderungen im Alltag als auch in der Wirtschaft gef{\"u}hrt hat. 
Durch die Digitalisierung mussten viele Branchen wie die Musik, Einzelhandel und die Logistik \& Versand Industrie umstrukturieren oder wurden wie zum Beispiel die Schreibmaschinenindustrie vollst{\"a}ndig beseitigt. 
Somit erscheinen t{\"a}glich technologische Innovationen, die im Internet, Fernsehen oder in Zeitschriften ver{\"o}ffentlicht werden. 
Ein Bereich sind die affektiven Technologien. 
Dieser Bereich hat sich als interdisziplin{\"a}res Forschungsfeld etabliert und untersucht die Interaktion zwischen Mensch und Maschine, wobei Emotionen im Mittelpunkt stehen und l{\"a}sst sich in zwei Systeme unterteilen. 
Zu einem die emotionssensitiven Technologien, womit Maschinen verstehen was Menschen f{\"u}hlen und zum anderen die ``Emotional Robotic''- Technologien, die einen Roboter menschen{\"a}hnlicher erscheinen lassen. 
Um die Forschung im Bereich emotionssensitiven Technologien voranzutreiben wurde ein Forschungsprojekt mit den Namen ``ELISE: Entwicklung von interaktiven und emotionssensitiven Lernsystemen zur Kompetenzerhaltung im Gesch{\"a}ftsprozessmanagement''  ins Leben gerufen (siehe Kapitel \ref{elise-subsec}).
Der Lehrstuhl Medizinische Informatik und Mikrosystementwurf entwickelt im Rahmen des Gesamtprojektes ein Sensorsystem, welches die Vital-, Elektroenzephalografie- und Elektrookulografiewerte aufzeichnet. 
Diese werden dann vom Lehrstuhl f{\"u}r Mustererkennung ausgewertet. 
Die Lerninhalte der Hauptanwendung des ELISE-Projekts werden daraufhin an Emotionen und Gem{\"u}tslagen der Lernenden wie Gl{\"u}ck, Langeweile, Frustration auf Basis von biomedizinischer Daten angepasst, um so den individuellen Erfolg des Lernenden zu erh{\"o}hen. 
Diese Projektarbeit befasst sich mit dem Entwurf eines kompakten mikrocontrollergest{\"u}tzten Systems zur Emotionserkennung in einer Virtual-Reality-Umgebung, Die Projektarbeit baut auf eine vorher am Lehrstuhl geschriebene Master Thesis  auf\cite{msckroenert} und ist eine Zusammenarbeit mit dem Lehrstuhl f{\"u}r Mustererkennung. 
Sie beinhaltet den Aufbau und die Programmierung des Mikrocontrollers, welches zur Kommunikation der verschiedenen biomedizinischen Sensoren dient, die Entwicklung der VR-Umgebung, die Schnittstellen zwischen Hardware und Software und die Speicherung der Rohdaten (Vital-, Elektroenzephalografie- und Elektrookulografiewerte). 
Diese Rohdaten wurden anhand von Messreihen an 88 Probanden gewonnen und wurden dem Lehrstuhl f{\"u}r Mustererkennung zur Verarbeitung {\"u}bergeben. 
Die Ergebnisse der Verarbeitung werden auch aufgef{\"u}hrt. 

