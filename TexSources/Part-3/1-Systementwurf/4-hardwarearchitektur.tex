\subsection{Hardwarearchitektur} \label{hardwarearchitektur-subsec}

In den nächsten Abschnitten werden die verwendeten Sensoren, sowie die dazugehörigen Messschaltungen näher erläutert. Im Rahmen dieser Projektgruppe wurden insgesamt drei Prototypen gefertigt. In einigen Fällen ist der Sensor, und die dazugehörige Schaltung  unverändert geblieben. Bei anderen Sensoren gab es Änderungen. In diesem Fall werden die Sensoren und Schaltungen für jeden Prototypen aufgeführt und erläutert.

Die Auswahl der zur Emotionsbestimmung nötigen Vitaldaten wurde im Rahmen der bereits erwähnten aufbauenden Masterarbeit von David Krönert bestimmt. Dies waren die Körpertemperatur, Sauerstoffsättigung im Blut, Blutvolumenpuls, Hautleitfähigkeit, Gehirnaktivität und die Augenbewegungen. Diese Auswahl von relevanten Vitaldaten wurde im leufe der Projektgruppe nicht mehr geändert. Einzig die genaue Art der Messung hat sich mit den unterschiedlichen Varianten des Messboards geändert. Im folgenden soll deshalb noch eine kurze Erklärung der einzelnen Sensoren erfolgen, und wie sich diese für unterschiedliche Prototypen geändert haben.  


% Unterkapitel
\subsubsection{GSR-Sensor} \label{gsr-1}

GSR ist eine Abkürzung für das englische Galvanic Skin Response und ist synonym mit der Abkürzung EDA (engl. Elektrodermal activity dt. Elektrodermale Aktivität). Es wird also mit dem GSR-Sensor die Hautleitfähigkeit zwischen zwei leitenden Elektroden gemessen. Für unsere Zwecke, also der Emotionserkennung, ist dies insofern relevant, da einige Emotionen durchaus Einfluss auf die Hautleitfähigkeit eines Probanden nehmen können, so wie zum Beispiel negative Emotionen wie Angst oder Stress die Schweißproduktion des Körpers beeinträchtigen können. Dadurch ändert sich dann natürlich auch die Leitfähigkeit der Haut.

Im wesentlichen wurde die Hautleitfähigkeit mittels zweier verschiedener Messeinrichtungen bestimmt. Bei den ersten beiden Prototypen war der GSR-Sensor im wesentlichen ein Spannungsteiler. Der genaue Aufbau kann der Abbildung X entnommen werden. 

%BILD

Um ein direktes einspeisen der Versorgungsspannung VCC auf den Probanden zu vermeiden wurde der Widerstand R6 (68kOhm) als Strombegrenzung eingeführt. 1 und 2 an JP4 sind die am Probanden befindlichen Elektroden, zwischen denen der Widerstand der Haut (R2) gemessen werden soll, P02 ist die zugehörige Spannung, und wird an den Analog-Digital-Wandler der Mikrocontrollers weitergegeben. C3 ist zum Abfangen von Spannungsspitzen da. Daraus ergibt sich folgende Gleichung für P02:

P02 = Vcc + R1/(R1 + R2)  %Formatieren



% Unterkapitel
\subsubsection{Temperatur-Senosr} \label{temp-1}








% Unterkapitel 
\subsubsection{Pulsoximeter} \label{pulsoximeter-1}

Die Auswahl des Verwendeten Pulsoximeters hatte sich im Laufe des Projektes nicht geändert, eine genaue Beschreibung ist dem Kaptiel 5.4.3 Pulsoximeter zu entnehmen.



% Unterkapitel 
\subsubsection{EEG} \label{eeg-subsubsec}

Die EEG-Schaltung hat sich im vergleich zum zweiten Prototypen nicht mehr Grundlegend geändert, es wurde lediglich eine Minimierung der Schaltung erreicht. eine genauere Beschreibung befindet sich in Kaptiel 11.2.4 EEG.





% Unterkapitel 
\subsubsection{EOG} \label{eog-1}








% Unterkapitel
\subsubsection{Datenübertragung} \label{datenuebertragung-subsubsec}








