\subsubsection{Festlegung der genutzten Hardware} \label{festlegung-subsubsec}

Für den zweiten Prototypen wurde der ursprüngliche Entwurf wieder komplett verworfen. Die wesentliche Neuerung im Vergleich zum vorherigen Prototypen bestand darin, dass die Analogen Signale EEG und EOG nicht mehr wie zuvor mit AD-Wandler und Operationsverstärker bestimmt wurden, sondern mit einem speziell dafür vorgesehenen Chip. Dem ADS1299 der Firma Texas-Instruments. Dieser Chip wurde für die Umwandlung von EEG-Signalen entwickelt, und ist somit in der Lage kleinste Spannungsschwankungen zu erkennen. Dazu hat jeder Kanal des integrierten AD-Wandlers zwei Eingänge, über die an einem Differenzverstärker der Spannungsunterschied zwischen zwei Elektroden gemessen wird. Dies ermöglicht eine Differenzmessung und auch eine Referenzmessung für die EEG-Signale. Dadurch wurde eine massive Platzersparnis im Vergleich zum ersten Prototypen erreicht, da die Verwendung von Operationsverstärkern komplett entfiel. Zudem wurde hier eine höher Auflösung erreicht, was allerdings die Vergrößerung der Datenpakte zur folge hatte, von 8-Bit im ersten Prototypen zu jetzt 24-Bit. Die Daten werden hier über eine SPI-Schnittstelle übertragen. Dadurch wurde ebenfalls ein Wechsel des Verwendeten Mikrocontrollers bedingt. Es galten hier die gleichen Kriterien wie auch zuvor. Letztendlich fiel die Wahl auf den Mikrocontroller ESP32 der Firma Espressif. Dieser verfügt über ausreichend Schnittstellen für I2C,SPI und UART. Außerdem ist hier auch ein ausreichend guter AD-Wandler für das analoge GSR-Signal vorhanden. Wegen der bereits integrierten Antenne wurde hier das Modell ESP-WROOM-32 gewählt. Zusätzlich zu Antenne ist die gesamte benötigte Peripherie ebenfalls schon enthalten, so musste nur ein Spannungspin an 3,3 Volt, mit einem Kondensator zur Spannungsglättung, angeschlossen werden. Dadurch wurde eine weiter Ersparnis in Hinsicht auf den Platzbedarf erzielt. Als letzte Verbesserung wurde hier noch ein Chip vom Typ FT232RL eingefügt, mit der eine Reibungslose Kommunikation, und dadurch Programmierung, zwischen der UART-Schnittstelle der Mikrocontrollers und eine USB-Schnittstelle ermöglicht wurde.




